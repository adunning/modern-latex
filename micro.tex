\chapter{Microtypography}

\introduce{Microtypography} is the use of small, subliminal tweaks to improve
the legibility of a document. It is
\begin{quote}
[\ldots]the art of enhancing the appearance and readability of a
document while exhibiting a minimum degree of visual obtrusion.
It is concerned with what happens between or at the margins of characters,
words or lines. Whereas the macro-typographical aspects of a document
(i.e., its layout) are clearly visible even to the untrained eye,
micro-typographical refinements should ideally not even be recognisable.
That is, you may think that a document looks beautiful, but you
might not be able to tell exactly why: good micro-typographic practice tries to
reduce all potential irritations that might disturb a reader.\punckern\endnote{%
R Schlicht,
\textit{The microtype package}
(v2.7a, January 14, 2018), 4}
\end{quote}

In \LaTeX{}, microtypography is enabled and controlled via the
\texttt{microtype} package.
Its use is automatic---for the vast majority of documents, you add
\begin{leftfigure}
\begin{lstlisting}
\usepackage{microtype}
\end{lstlisting}
\end{leftfigure}
to your preamble and carry on---but let's take a brief look at what the package
actually does.

\section{Character protrusion}

By default, \LaTeX{} justifies lines between perfectly straight
left and right margins.
This is the obvious choice,
but falls victim to an annoying optical illusion:
lines ending in small punctuation---like a period or a
comma---seem shorter than lines that
don't.\punckern\footnote{Many other optical illusions are
relevant to typography. For example, if a circle, a square, and a triangle
of equal heights are placed next to each other,
the circle and triangle look smaller than the square.
For this reason, circular or pointed characters (like O and A) must
be made slightly taller than ``flat'' ones (such as H and T) for all
to appear the same height.\punckern\endnote{%
Jost Hochuli, \textit{Detail in typography} (Éditions~B42, 2015), 18--19}}
\texttt{microtype} \introduce{protrudes} punctuation and other smaller
characters into the margins slightly to compensate.

\section{Font expansion}

To assist \LaTeX's justification algorithm,
\texttt{microtype} can
stretch or shrink characters horizontally if
this helps produce a better layout, e.g.,
one with more even spacing, or fewer lines which end in hyphens.
You might think that distorting character shapes this way would be immediately
noticeable to the reader,
but you're reading an entire book that is justified this way!
\texttt{microtype} applies font expansion \emph{very} slightly---by default,
character widths are changed by no more than 2\%.\punckern\footnote{Of course,
you can use package options to change this limit,
or disable the feature entirely.}

Note that this feature isn't currently available with \XeLaTeX{}.
You'll need to use \LuaLaTeX{} if you want to take advantage of it.

\section{Other stuff}

TODO other features and why you don't need them
