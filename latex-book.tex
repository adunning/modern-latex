% Build this document with LuaTeX, a modern Unicode-aware LaTeX engine
% that uses system TTF and OTF font files.
% This is needed for the fontspec, microtype, and nolig packages.
%
% We're using KOMA Script to hand-tune footnotes and TOC appearance.
% It should be available in your texlive distribution,
% which is how most distros package LaTeX.
\documentclass[fontsize=10pt, numbers=endperiod, openany]{scrbook}

% Margins: see http://practicaltypography.com/page-margins.html and
% http://practicaltypography.com/line-length.html
% We're aiming for 80-ish characters per line.
\usepackage[paperwidth=6.25in, paperheight=9.25in,
            layoutwidth=6in, layoutheight=9in,
            layouthoffset=0.125in, layoutvoffset=0.125in,
            inner=0.5in,outer=0.75in,top=0.5in,bottom=0.5in,
            includefoot,
            showcrop
           ]{geometry}

% Font specification.
% Use Equity for the body text,
% Roboto for sans serif,
% and mononoki for monospace text.
% Feel free to pick your own fonts or comment these lines out to use TeX's
% traditional Computer Modern.
\usepackage{fontspec}
\usepackage[fleqn]{amsmath}
%\setmainfont[Ligatures=TeX,
%             Numbers=Lowercase,
%             SmallCapsFont={Equity Caps A},
%             SmallCapsFeatures={StylisticSet=10}, % Set everything in \textsc{} as small caps
%             StylisticSet=01, % Slightly smaller quotes, asterisks, etc.
%            ]{Equity Text A}

% Help, I've fallen down a rabbit hole.
% Define better kerns following f
\directlua
{
  fonts.handlers.otf.addfeature {
    name = "morekerns",
    type = "kern",
    data = {
      ["f"] = {
        ["!"] = 150,
        ["T"] = 120
      },
    },
  }
}

\setmainfont[Ligatures=TeX, Numbers={Proportional,Lowercase},
             BoldFont=AGaramond Pro Semibold,
             BoldItalicFont=AGaramond Pro Semibold Italic,
             RawFeature=+morekerns
            ]{Adobe Garamond Pro}
\setsansfont[Ligatures=TeX,
             Scale=MatchUppercase,
             Style=Alternate, % Straight-legged R
             UprightFont = *-55Rg,
             ItalicFont = *-56It,
             BoldFont = *-65Md,
             BoldItalicFont = *-66MdIt
             ]{NHaasGroteskTXPro}
\setmonofont[Scale=MatchLowercase]{mononoki}

% We'll be using this quite a bit:
\newfontfamily{\lm}[%
    Ligatures=TeX,
    SmallCapsFont = * Caps
]{Latin Modern Roman}

\usepackage{polyglossia}
\setdefaultlanguage[variant=american]{english}
\setotherlanguage{vietnamese}

\usepackage{blindtext}

\usepackage{microtype} % Font expansion, protrusion, and other goodness

% Disable ligatures across grapheme boundaries
% (see the package manual for details.)
\usepackage[american]{selnolig}

% Use symbols for footnotes, resetting each page
\usepackage[perpage,bottom,symbol*]{footmisc}

% Left flush footnotes. See the KOMA Script manual.
\deffootnote[1em]{1em}{1em}{\thefootnotemark}
% Set the width of the rule separating body text and footnotes
\setfootnoterule{0.7\textwidth}

% Like many fonts, Equity's asterisk is already set in a "superscripted" form.
% Superscripting *that* makes it annoyingly small.
% To fix this, we have to redefine footnote marks so that they aren't superscript,
% then raise all the other symbols.
%
% Feel free to remove this if your body type doesn't have this peculiarity,
% but unfortunately many do.
% See http://tex.stackexchange.com/a/16241
%
% We use the Unicode symbols themselves (instead of \dagger, \ddagger, \P, etc.)
% because the latter fall back to Computer Modern/Latin Modern in some cases,
% (e.g., if you're using mathastext instead of unicode-math).
% Alternatively, you could use \textdagger, \textddagger, etc.,
% but this seems more concise.
\DefineFNsymbols*{tweaked}{%
    {*}%
    {\textsuperscript†}%
    {\textsuperscript‡}%
    {\textsuperscript{◊}}%
    {\textsuperscript{¶}}%
    {**}%
    {\textsuperscript{††}}%
    {\textsuperscript{‡‡}}%
}
\setfnsymbol{tweaked}
\deffootnotemark{\thefootnotemark}

\DeclareTOCStyleEntry[%
    beforeskip=5pt,
    entrynumberformat = \addfontfeature{Numbers={Tabular,Uppercase}},
    pagenumberformat = \addfontfeature{Numbers={Tabular,Uppercase}},
    linefill = \TOCLineLeaderFill
]{tocline}{chapter}

% Don't use a sans font for description labels.
\addtokomafont{descriptionlabel}{\rmfamily}
\setkomafont{disposition}{\rmfamily}
\addtokomafont{chapter}{\addfontfeature{Numbers=Uppercase}}
\setkomafont{section}{\Large\itshape}
\renewcommand*\thesection{\upshape\arabic{section}}

%\setcapwidth[l]{.8\textwidth}
%\setcapmargin{0pt}
\setkomafont{caption}{\sffamily\footnotesize}
\setkomafont{captionlabel}{\sffamily\footnotesize}
\renewcommand*{\figureformat}{}
\renewcommand*{\tableformat}{}
\renewcommand*{\captionformat}{}

% Use uppercase numbers for numbered lists.
% (We're using lowercase ones for the body text.)
% See http://tex.stackexchange.com/a/133186
\usepackage{enumitem}
\setlist[enumerate]{font=\addfontfeatures{Numbers=Uppercase}}

% Custom footer
\usepackage{scrlayer-scrpage}
\clearpairofpagestyles
\pagestyle{scrheadings}
\setkomafont{pagefoot}{\upshape}
\cefoot*{\thepage} % \, is a TeX primitive for a thin space.
\cofoot*{\thepage} % \, is a TeX primitive for a thin space.

\usepackage{graphicx}

\newcommand{\codesize}{\fontsize{10pt}{12pt}}

\usepackage{changepage} % For adjustwidth

\usepackage{mflogo} % for METAFONT
\usepackage{metalogo} % for \LuaLaTeX

\usepackage{endnotes}
% Endnotes _mostly_ works... with Koma Script, no less.
% I suppose I can only grumble a little about the hackery below.
\renewcommand\enoteheading{\chapter{Notes}}
\renewcommand\enoteformat{\leavevmode\llap{\makeenmark}}
% OTF goodness...
\renewcommand\makeenmark{{\addfontfeature{VerticalPosition=Superior}\theenmark}}

\usepackage{listings}
\lstset
{
    language=[LaTeX]TeX,
    breaklines=false,
    basicstyle=\ttfamily\small,
    keywordstyle=\ttfamily\small,
}

\newenvironment{leftfigure}
  {\par\vspace{0.5\baselineskip minus 0.25\baselineskip}\begin{adjustwidth}{3em}{0pt}}
  {\end{adjustwidth}\vspace*{0.3\baselineskip minus 0.2\baselineskip}}

\title{Modern LaTeX}
\author{Matt Kline}
\date{\today}

% Custom footer
% Hyperlinks
\usepackage[unicode,pdfusetitle,hidelinks]{hyperref}

% Use \punckern to overlap periods, commas, and footnote markers
% for a tighter look.
% Care should be taken to not make it too tight - f" and the like can overlap
% if you're not careful.
\newcommand{\punckern}{\kern-0.3ex}
% For placing commas close to, or under, quotes they follow.
% We're programmers, and we blatantly disregard American typographical norms
% to put the quotes inside, but we can at least make it look a bit nicer.
\newcommand{\quotekern}{\kern-0.5ex}


% Create an unbreakable string of text in a monospaced font.
% Useful for `command --line --args`
\newcommand{\monobox}[1]{\mbox{\texttt{#1}}}

% C++ looks nicer if the ++ is in a monospace font and raised a bit.
% Also, use uppercase numbers to match the capital C.
\newcommand{\plusplus}{\raisebox{0.2ex}{++}}
\newcommand{\cpp}[1]{C\kern-0.1ex\plusplus{\addfontfeature{Numbers=LowercaseOff}#1}}
\newcommand{\clang}[1]{C{\addfontfeature{Numbers=LowercaseOff}#1}}
\newcommand{\csharp}{C\raisebox{0.25ex}{\#}}

\newcommand{\fig}[1]{Figure~\ref{#1}}

% Italicize new terms
\newcommand{\introduce}[1]{\textit{#1}}

% Letterspace acronyms a bit.
\newcommand{\acronym}[1]{\textsc{\addfontfeature{LetterSpace=8}#1}}

% See http://tex.stackexchange.com/a/68310
\makeatletter
\let\runauthor\@author
\let\rundate\@date
\let\runtitle\@title
\makeatother

% Spend a bit more time to get better word spacing.
% See http://tex.stackexchange.com/a/52855/92465
\emergencystretch=1em

\begin{document}
\fontsize{10pt}{13pt}\selectfont

\frontmatter
\setcounter{secnumdepth}{0}
\setlength\parindent{0pt}

% Custom title instead of \maketitle
\pagenumbering{gobble}
\vspace*{1.5in}
\begin{center}
\fontsize{0.5in}{0.7in}\selectfont
Modern

\fontsize{1in}{1.1in}\selectfont
\LaTeX
\vfill
\LARGE
Matt Kline
\end{center}
\clearpage

\null
\vfill
Some crappy draft, typeset \today.
\vspace*{0.5in}

The author apologizes for typos, misattributions, or any other flubs,
and welcomes you to point them out on this book's Github
repository at \\
\url{https://github.com/mrkline/latex-book}, or via email to \\
\texttt{matt <at> bitbashing.io}

The author does not have a checking account with the Bank of San Serriffe,
but will happily purchase a beverage of your choice the next time we meet.

\vspace*{0.5in}
{\addfontfeature{Numbers={Proportional,Uppercase}}
Copyright © 2018 \\
by \runauthor
\bigskip

This book is licensed under the \\
Creative Commons Attribution-ShareAlike~4.0 International License. \\
The full text of the license is available at \\
\url{https://creativecommons.org/licenses/by-sa/4.0/legalcode}
}
\clearpage

\vspace*{1in}
{\itshape%
%\noindent To Donald, Leslie, \\
%Ellen, Erik, Jost, Matthew, \& Robert
%\bigskip

To Max, who once told me about a cool program he was using to type up
his college papers.
}
\cleardoublepage

\pagenumbering{roman}
\tableofcontents

\mainmatter
\setlength\parindent{1.5em}

\pagenumbering{arabic}
\setcounter{page}{1} % Restart page numbering after the ToC.
\cleardoublepage

\chapter{\texorpdfstring{\LaTeX}{LaTeX}?}

\LaTeX{} is a program\footnote{And a markup language,
and (sort of) a programming language, and a bit of a cult.
But that's all for later.}
for crafting written documents, like papers, presentations,
or even the book you're reading right now.
In some sense, it's an alternative to the likes of Microsoft Word,
Apple Pages, Google Docs, or LibreOffice Writer.

Unlike these other applications, which work on the principle of
\introduce{What You See Is What You Get} \acronym{(wysiwyg)},
\LaTeX{} documents are built from a ``plain'' text file,
using \introduce{markup} to specify how things should look.
When you are ready, \LaTeX{} generates your document from this file
based on the formatting rules you gave it.
If you've done any web development, this is a similar process---just
as \acronym{html} and \acronym{css} describe the page you want
browsers to draw, your markup describes the appearance of your
document to \LaTeX.

\begin{leftfigure}
\begin{lstlisting}
Unlike these other applications, which work on the principle of
\introduce{What You See Is What You Get} \acronym{(wysiwyg)},
\LaTeX{} documents are built from a ``plain'' text file,
using \introduce{markup} to specify how things should look.
When you are ready, \LaTeX{} generates your document from this file
based on the formatting rules you gave it.
If you've done any web development, this is a similar process---just
as \acronym{html} and \acronym{css} describe the page you want
browsers to draw, your markup describes the appearance of your
document to \LaTeX.
\end{lstlisting}
\captionof{figure}{The \LaTeX{} markup for the above paragraph}
\end{leftfigure}

This might seem alien to you if you've never worked with a markup system before.
However, it comes with a few advantages:
\begin{enumerate}
\item You can handle a document's contents and its presentation separately.
    At the beginning of each document,
    you describe the design you want.
    From there, \LaTeX{} handles the rest for you,
    keeping fonts, sizes, line heights,
    and other layout considerations consistent across your whole text.
    Compare this to a \acronym{wysiwyg} system,
    where you must constantly concern yourself with your work's appearance
    as you write.
    If you changed the look of a caption,
    did you make sure to find all the other captions and do the
    same?\footnote{This isn't to say that it's impossible
    to create nice, consistent documents with \acronym{wysiwyg} tools.
    Many of them have gotten much better at automating these sorts of
    tasks in recent years.}
    If the program formats something in a way you don't like,
    how hard is it to fix?%\footnote{I spent far too much of my childhood
    %fighting with Word about how it wrapped text around images.}

\item You can define your own rules, then tweak them to immediately change
    everywhere they're used in your document.
    For example, the \verb|\introduce| and \verb|\acronym| commands you saw above
    are my own creations. The former \introduce{italicizes} text, and
    the latter sets words in \acronym{small caps} with a bit of extra
    \,\textsc{\addfontfeature{LetterSpace=15}letterspacing}\, so the characters
    don't look \textsc{too crowded}.
    If I wake up tomorrow and decide to introduce new terms
    \textbf{\itshape with this look} and set acronyms
    {\small\addfontfeature{LetterSpace=8} LIKE THIS},
    I just change the two lines that define those commands
    to update every place in the book that uses them.
\item In \LaTeX, the document's structure is immediately visible
    and can be easily copied.
    In \acronym{wysiwyg} systems, it is often unobvious
    how certain formatting was produced
    or how to replicate it.
\item Since the document source is plain text,
    \begin{itemize}
    \item It can be read and understood with any text editor.
    \item Content can be easily generated programmatically.
    \item Changes are easily tracked with version control software.
    \end{itemize}
\end{enumerate}

This is all well and good,
but misses the main selling point of \LaTeX.
You should try it because you care about\ldots

\chapter{Typography}
\label{typography}

Modern life is a constant battle for your time---every day,
dozens of ads, apps, emails, sites, and texts fight
for a few short minutes of it.
To put ideas into the world,
nothing is more important than catching and holding
the attention of your audience.
Typography is a tool to do so---a good design doesn't ``look nice''
only for the sake of art---it draws readers in.\punckern\endnote{Matthew Butterick,
``Typography for Docs''
(presented at the Write The Docs Conference, April 8, 2013),
\url{https://www.youtube.com/watch?v=8J6HuvosP0s}}
%Matthew Butterick's \textit{Practical Typography} calls it
%``the visual component of the written word.\quotekern''
It sets their expectations and establishes a subliminal ``brand'' for your
work.\punckern\endnote{Erik Spiekermann, ``Type is Visible Language''
(presented at Beyond Tellerrand, Düsseldorf, Germany, May 19--21, 2014),
\url{https://www.youtube.com/watch?v=ggQpDu63kk0}}
It's the \emph{how} of text.
\begin{leftfigure}
\fontspec{TeX Gyre Termes}\fontsize{12pt}{24pt}\selectfont\raggedright
Typography is why this reminds you of the terrible essays
you wrote in school.
Would you like to read an entire book this way?
\end{leftfigure}
It is why
\begin{leftfigure}
\noindent\fontspec{FuturaCon-ExtBol}\Large DO IT LATER
\end{leftfigure}
seems oddly reminiscent of a certain shoe company's advertising,
or how
\begin{center}
\fontspec[Ligatures=TeX, Scale=MatchLowercase]{Futura-Med}
\noindent MEN FROM THE PLANET EARTH \\
FIRST SET FOOT UPON THE MOON \\
JULY 1969, A.~D.
\end{center}
Typography shows people
what you have to say before they read a single word.

This is where \LaTeX{} shines, and
unless you want to give Adobe large sums
of money for InDesign or InCopy,
you won't find any better typesetting software.
By carefully handling subtle details---like how words are spaced,
hyphenated, and arranged into lines---\LaTeX{} provides high-quality layout
with relatively little effort from you, the author.
Modern versions can also take advantage of new\footnote{By new,
I mean ``from the mid-1990s''\quotekern, but web browsers and desktop publishing
software are only just starting to catch up.} advances in computer typography,
giving you the same tools available to professional designers and publishers.

\chapter{Another Guide?}

One might wonder why the world needs another \LaTeX{} guide.
After all, it's been out for more than 30 years.
A quick Amazon search finds nearly a dozen books on the topic.
There are plenty of great resources online.

Unfortunately, most guides have two fatal flaws: they are long,
and they are old.
The first is anathema to newcomers---if somebody asks about \LaTeX{},
throwing a 200+ page book at them isn't an encouraging start.
The second matters because of how much the world of computer typesetting has
changed since 1986.

When \LaTeX{} was first released that year, none of the publishing technologies
we use today existed.
Adobe wouldn't introduce their Portable Document Format for another seven years,
and home computers wouldn't support it ubiquitously for several more.
Digital typography was a new field, and desktop publishing was \emph{just}
getting off the ground.\punckern\endnote{\textit{Graphic Means:
A History of Graphic Design Production}, directed by Briar Levit (2017)}
This shows---badly---in most \LaTeX{} guides.
If you look for instructions to change your document's font,
you'll likely get swamped with bespoke nonsense.\punckern\footnote{%
Take all criticisms of \LaTeX's past here with a grain of
salt. After all, the fact that all of the technology around it became
obsolete---multiple times---is a testament to its staying power.}

The good news is that  \LaTeX{} has improved by leaps and bounds in recent years.
It's time for a guide that doesn't weigh you down with decades of legacy.
It's also time for one that doesn't try to be some comprehensive reference.
After all, you're a smart, resourceful individual who knows how to use a search
engine.
This book will:
\begin{enumerate}
\item Teach you the very basics of using \LaTeX.
\item Point you to places where you can learn more.
\item Show you how to use modern technologies like OpenType and microtypography
    to create professional-quality documents.
\item End promptly thereafter.
\end{enumerate}
Let's begin.


\chapter{Installation}
\label{installation}

You install \LaTeX{} on your computer by downloading a \introduce{distribution},
which comes with:
\begin{enumerate}
\item \LaTeX, the program---the thing that turns text files into typeset
    documents.\footnote{We'll actually be installing multiple \LaTeX{} programs,
    but we're getting to that.}
\item A common set of \LaTeX{} \introduce{packages}.
    Packages are pieces of code that can be placed in your documents to
    do all sorts of things, like change their style or add new commands.
    We'll see lots of them in action throughout this book.
\item Miscellaneous tools, such as editors.
\end{enumerate}
Each major operating system has its own \LaTeX{} distribution:
\begin{description}
\item[Mac OS] has Mac\TeX. Grab it from \http{www.tug.org/mactex}
    and follow the instructions there to install it.

\item[Windows] has Mik\TeX.
    Install it from \https{miktex.org/download}.
    Mik\TeX{} offers the unique ability to automatically download
    additional packages if your document uses one that it can't find
    on your computer.

\item[Linux and BSD] use \TeX{} Live.
    Like most software, this is provided through your
    \acronym{os}'s package manager.
    Linux distributions usually contain a \texttt{texlive-\allowbreak full}
    or \texttt{texlive-\allowbreak most} package that installs everything
    you need.\punckern\footnote{%
    If you'd rather keep the install size down,
    Linux distributions usually break \TeX{} Live into multiple distro packages.
    Look for ones with names like
    \texttt{texlive-\allowbreak core}, \texttt{texlive-\allowbreak luatex}
    and \texttt{texlive-\allowbreak xetex}.
    As you work with \LaTeX, you may need less-common packages,
    which usually have names like \texttt{texlive-\allowbreak latexextra},
    \texttt{texlive-\allowbreak science}, and so on.
    Of course, all of this may vary from one Linux distribution to another.}
\end{description}

\section{Editors}

Since \LaTeX{} source files are regular text files,
you're free to create them with the usual choices of Vim, Emacs,
Sublime, Notepad\plusplus, and so on.\punckern\footnote{If you have never used
any of these, try a few.
They're popular among programmers and other folks who shuffle text around
screens all day. Just don't use Notepad. Life is too short.}
However, there are also editors designed specifically for \LaTeX{},
which often come with their own built-in \acronym{pdf} viewer.
You can find a fairly comprehensive list on the \LaTeX{} Wikibook,
in its installation chapter. (See Appendix~\ref{resources}.)

\section{Online options}

If you'd rather not install \LaTeX{} on your computer,
you can try online editors such as Share\LaTeX{} and Overleaf.
This book won't focus on these web-based tools,
but almost all of the same basics apply.
(Of course, you have less control over certain aspects of online versions,
such as available fonts, the version of \LaTeX{} that's used, and so on.)


\chapter{Hello, \texorpdfstring{\LaTeX}{LaTeX}!}

Now that you have a \LaTeX{} distribution installed,
let's try it out.
Open up your editor of choice and save the following as \texttt{hello.tex}:
\begin{leftfigure}
\begin{lstlisting}
\documentclass{article}
% Say hello
\begin{document}
Hello, World!
\end{document}
\end{lstlisting}
\end{leftfigure}
We then run this file through \LaTeX{} (the program)\footnote{Not to be
confused with \LaTeX{} the lunchbox, \LaTeX{} the breakfast cereal,
or \LaTeX{} the flamethrower. The kids love this stuff!}
to get our document.
The installation places several different versions---or
\introduce{engines}---on your machine,
but for the entire book, we'll always use \LuaLaTeX{} or \XeLaTeX.
These are the newest engines available---see Appendix~\ref{history} for an
explanation of how they differ from the others.

If you are using a \LaTeX{}-specific editor, it may contain a drop-down menu
or some other configuration to select the engine you'd like to use,
as well as a button to generate your document.
Otherwise, from a terminal,\punckern\footnote{How to work a terminal emulator,
making sure the newly-installed \LaTeX{} programs are in your \texttt{PATH},
and so on are all outside the scope of this book.
As is tradition, the leading dollar sign in examples just denotes a console
prompt, and shouldn't actually be typed.}
run the following:
\begin{leftfigure}
\begin{lstlisting}
$ xelatex hello.tex
\end{lstlisting}
\end{leftfigure}
Feel free to use \texttt{lualatex} instead---there are a few differences
between the two, but either is fine for now.
With luck, you should see some output that ends in a message like:
\begin{leftfigure}
\begin{lstlisting}
Output written on hello.pdf (1 page).
Transcript written on hello.log.
\end{lstlisting}
\end{leftfigure}
and in your current directory, a newly minted \texttt{hello.pdf}.
Open it up and you should find a page with this at the top:
\begin{leftfigure}
\lm Hello, World!
\end{leftfigure}
Congrats,
you just created your first document!

Let's unpack what we did.
All \LaTeX{} documents begin with a \verb|\documentclass| declaration,
which picks a style to use.
Many classes are available---and you can even create your own---but common
ones include \texttt{article}, \texttt{report}, \texttt{book},
and \texttt{beamer}.\punckern\footnote{This last one is for slideshows.
Kind of an odd name, no?}
For your average document, \texttt{article} is probably a good choice.
Next is, \verb|% Say hello|.
This is a \introduce{comment}---anything placed after a percent symbol on a line
is ignored by the engine,
so we can use \verb|%| to leave notes about what's going on to anybody reading
the document's source.\punckern\footnote{Including, perhaps most importantly,
a confused version of your future self!}
Finally, we use \verb|\begin{document}| to tell \LaTeX{} that what follows
is our actual contents,
and we use \verb|\end{document}| to state that we are finished.

So far, so good.
Let's cover some more basics.

\section{Spacing}

\LaTeX{} generally handles inter-word spacing for you, regardless of how many
spaces\footnote{Or tabs} you type. For example,
\begin{leftfigure}
\begin{lstlisting}
The number  of   spaces    between words doesn't   matter. The same
is true for sentences.

An empty line starts a new paragraph.
\end{lstlisting}
\end{leftfigure}
yields
\begin{leftfigure}
\lm The number  of   spaces    between words doesn't   matter. The same
is true for sentences.

An empty line  starts a new paragraph.
\end{leftfigure}
Notice that \LaTeX{} is automatically following typographic
conventions, such as indenting new paragraphs and leaving more space after a
period than it leaves between words.
One quirk to be aware of is that comments ``eat'' all of the leading
space on the subsequent line, such that
\begin{leftfigure}
\begin{lstlisting}
This% weird, right?
  is strange
\end{lstlisting}
\end{leftfigure}
gives
\begin{leftfigure}
This% weird, right?
  is strange
\end{leftfigure}

\section{Commands}

\LaTeX{} has various commands to issue instructions to
the engine, and you can define your own as well.
Their names always begin with a backslash (\,\texttt{\textbackslash}\,),
contain only letters, and are case-sensitive.\punckern\footnote{%
% \verb doesn't play nicely in footnotes, so...
\texttt{\textbackslash foo}
is different from \texttt{\textbackslash Foo}, for example.}
Some commands require parameters, or \introduce{arguments}---\verb|\documentclass|,
for example, needs to know which document class we want.
These arguments are enclosed in subsequent pairs of braces,
so if some command needed two arguments, we would type:
\begin{leftfigure}
\begin{lstlisting}
\somecommand{argument1}{argument2}
\end{lstlisting}
\end{leftfigure}
Many commands also take optional arguments, which are enclosed in square
brackets and precede the mandatory ones.
Say you want to inform \LaTeX{} that your document will be printed as
double-sided pages.\punckern\footnote{\texttt{twoside} introduces commands
that only make sense in the context of double-sided printing,
such as ones that skip to the start of the next odd page.
It also allows you to have different margins for even and odd pages,
which is useful for texts like this book.}
This is done with an optional argument to \verb|\documentclass|:
\begin{leftfigure}
\begin{lstlisting}
\documentclass[twoside]{article}
\end{lstlisting}
\end{leftfigure}

Other commands take no arguments at all, such as \verb|\LaTeX|,
which prints the \LaTeX{} logo.
Know that these commands consume any space that follows them.
For example,
\begin{leftfigure}
\begin{lstlisting}
\LaTeX is great, but it can be strange sometimes.
\end{lstlisting}
\end{leftfigure}
will give you
\begin{leftfigure}
\lm \LaTeX is great, but it can be strange sometimes.
\end{leftfigure}
You can fix this by adding an empty pair of braces to the command.
Of course, the braces aren't needed if there is no space to preserve:
\begin{leftfigure}
\begin{lstlisting}
Let's learn \LaTeX! \LaTeX{} is a powerful tool,
but a few of its rules are a little weird.
\end{lstlisting}
\end{leftfigure}
gets us
\begin{leftfigure}
\lm Let's learn \LaTeX! \LaTeX{} is a powerful tool,
but a few of its rules are a little weird.
\end{leftfigure}

\section{Special characters and line breaks}

Some characters have special meanings in \LaTeX.
We saw above, for example, that \verb|%| starts a comment
and \verb|\| starts a command.
The full list of special characters is:
\begin{leftfigure}
\begin{lstlisting}
# $ % ^ & _ { } ~ \
\end{lstlisting}
\end{leftfigure}
All have corresponding commands
if you want to make them appear in your document:
\begin{leftfigure}
\begin{lstlisting}
\# \$ \% \^{} \& \_ \{ \} \~{} \textbackslash
\end{lstlisting}
\end{leftfigure}
will produce
\begin{leftfigure}
\lm \# \$ \% \^{} \& \_ \{ \} \~{} \textbackslash
\end{leftfigure}
Regardless of what comes after them, you must always add braces to
the caret (\,\texttt{\^{}}\,) and tilde (\,\~{}\,).
This is a relic from days when these characters were used to produce
\introduce{diacritical marks}:
once upon a time, users had to typeset ``jalapeño'' as
\verb|jalape\~no|.
Today, we just type \texttt{ñ} into our source
file.\punckern\footnote{The ease with which you can do this depends
on the keyboard you're using, the language settings in your \acronym{os},
and your editor.
Later on, we'll talk much more about non-English languages and Unicode fun.}

If you're wondering why we print \texttt{\textbackslash} with
\verb|\textbackslash| instead of just \verb|\\|,
it's because the latter is the command to force a line break.
\begin{leftfigure}
\begin{lstlisting}
Give me \\
a brand new line!
\end{lstlisting}
\end{leftfigure}
obeys:
\begin{leftfigure}
\lm Give me \\
a brand new line!
\end{leftfigure}
Use this power judiciously---deciding how to break paragraphs into lines
automatically is one of \LaTeX{}'s best skills.

\section{Environments}

We often control our document by organizing text into \introduce{environments}.
These always start with \verb|\begin{name}| and conclude with \verb|\end{name}|,
where \texttt{name} is that of the environment.
Take \texttt{quote}, which adds additional spacing on both sides of a block
quotation:
\begin{leftfigure}
\begin{lstlisting}
Donald Knuth once wrote,
\begin{quote}
We should forget about small efficiencies,
say about 97\% of the time:
premature optimization is the root of all evil.
Yet we should not pass up our opportunities in that critical 3\%.
\end{quote}
\end{lstlisting}
\end{leftfigure}
gives us
\begin{leftfigure}
\lm
Donald Knuth once wrote,
\begin{quote}
We should forget about small efficiencies,
say about 97\% of the time:
premature optimization is the root of all evil.
Yet we should not pass up our opportunities in that critical 3\%.
\end{quote}
\end{leftfigure}

\section{Groups and command scope}
Some commands change how \LaTeX{} typesets the following text.
\verb|\itshape|, for example, \textit{italicizes} everything that comes after it.
If we want to limit a command's effect to a certain region, we can surround it
with braces.
\begin{leftfigure}
\begin{lstlisting}
{\itshape Sometimes we want italics}, but only sometimes.
\end{lstlisting}
\end{leftfigure}
is set as
\begin{leftfigure}
\lm {\itshape Sometimes we want italics}, but only sometimes.
\end{leftfigure}
The text within the braces forms a \introduce{group},
and all commands issued inside the group only take effect until it ends.
Environments also also form implicit groups:
\begin{leftfigure}
\begin{lstlisting}
\begin{quote}
\itshape If I italicize a quote, the following text will
use upright type again.
\end{quote}
See? Back to normal.
\end{lstlisting}
\end{leftfigure}
produces
\begin{leftfigure}
\lm
\begin{quote}
\itshape If I italicize a quote, the following text will
use upright type again.
\end{quote}
See? Back to normal.
\end{leftfigure}
One final use of groups is to get around the spacing oddities of zero-argument
commands: some prefer \verb|{\LaTeX}| over \verb|\LaTeX{}|.


\appendix

\chapter{A Brief History of \texorpdfstring{\LaTeX}{LaTeX}}

\label{history}

Donald Knuth is celebrated among programmers as
the man who coined the term \emph{analysis of algorithms}
and pioneered many computer science fundamentals we use today.
Knuth is perhaps most famous for his ongoing magnum opus,
\textit{The~Art of Computer Programming}.

When the first volume of \acronym{taocp} was released in 1968,
it was printed the same way most books had been since the turn of the century:
with \introduce{hot metal} type.
Letters were cast from molten lead,
then arranged into lines.
These lines were clamped together to form pages,
which were inked and pressed against paper.

By March of 1977, Knuth was ready for a second run of \acronym{taocp}, volume~2,
but he was horrified when he received the proofs.
Hot metal typesetting was expensive, complicated, and time-consuming,
so publishers had replaced it with phototypesetting,
which works by projecting images of characters onto film.
The new technology, while much cheaper and faster,
didn't provide the quality Knuth
expected.\punckern\endnote{Knuth, \textit{Digital Typography} (Stanford, 1999), 3--5}

The average author would have resigned themselves to this change and moved on,
but Knuth took great pride in his books' appearances,
especially for their mathematics.
Around this time, he also discovered the growing field of digital typesetting,
where glyphs are built from tiny dots,
packed together at over 1,000 per inch.
Inspired,
Knuth set off on one of the greatest yak shaves\footnote{Programmers
call seemingly unrelated work needed to solve their main problem
``yak shaving''\quotekern. The phrase is thought to originate from an episode
of \textit{The Ren~\&~Stimpy Show}.\punckern\endnote{``yak shaving''\quotekern,
\textit{The Jargon File},
\href{http://www.catb.org/~esr/jargon/html/Y/yak-shaving.html}%
{\texttt{www.catb.org/\~{}esr/jargon/html/Y/yak-shaving.html}}}}
of all time.
For years, he paused work on his books to create his own
typesetting system.
When the dust settled in 1978, Knuth had the first version of
\TeX.\punckern\footnote{The name ``\TeX{}'' comes from the Greek
{\fontspec[Scale=MatchLowercase]{NotoSerif-Medium}τέχνη},
meaning \introduce{art} or \introduce{craft}.\punckern\endnote{Knuth,
\textit{The \TeX book}, 1}}

It's hard to appreciate how much of a revolution \TeX{} was,
especially looking back from a time where anybody with a copy
of Word can be their own desktop publisher.
Adobe's \acronym{pdf} wouldn't exist for another decade, so Knuth
and his graduate students devised their own device-independent format,
\acronym{dvi}.
Scalable fonts were uncommon, so he created \MF{} to rasterize glyphs
into dots on the page.
Perhaps most importantly, Knuth and his students designed algorithms
to automatically hyphenate and justify text into
beautifully-typeset paragraphs.\punckern\footnote{These same algorithms went
on to influence the ones Adobe uses in its software today.\punckern\endnote{%
Several sources (\http{www.tug.org/whatis.html},
\https{tug.org/interviews/thanh.html},
\http{www.typophile.com/node/34620})
mention \TeX's influence on the \textit{hz}-program by Peter Karow
and Hermann Zapf, thanks to via Knuth's collaborations with Zapf.
\textit{hz} was later acquired by Adobe and used
when creating InDesign's paragraph formatting systems.}}

\LaTeX{}, short for Lamport~\TeX{}, was later developed by Leslie Lamport
as a set of commands for common document layouts.
It was introduced in 1986 with his guide,
\textit{\LaTeX: A~Document Preparation System}.
Other typesetting systems based on \TeX{} also exist,
the other most popular today being Con\TeX{}t.

Development continues,
both in the form of user-provided packages for \TeX{} and \LaTeX{},
and as improvements to the \TeX{} typesetting program itself.
There are four versions, or \introduce{engines}:
\begin{description}
\item[\TeX] is the original system by Donald Knuth.
Knuth stopped adding features after version 3.0 in March~1990,
and all subsequent releases have contained only bug fixes.
With each release, the version number asymptotically approaches $\pi$
by adding an additional digit.
The most recent version, 3.14159265, came out in January~2014.

\item[pdf\TeX] is an extension of \TeX{} that provides direct \acronym{pdf}
    output (instead of \TeX's \acronym{dvi}),
    native support for PostScript
    and TrueType fonts,
    and micro-typographic features discussed in \chapref{microtype}.
    It was originally developed by
    Hàn Thế Thành
    as part of his PhD thesis
    for Masaryk University in Brno, Czech Republic.\punckern\endnote{%
    Hàn Thế Thành,
    \textit{Micro-typographic extensions to the \TeX{} typesetting system}
    (Masaryk University Brno, October 2000)}

\item[\XeTeX] is a further extension of \TeX{} that adds native support for
    Unicode and OpenType.
    It was originally developed by Jonathan Kew in the early 2000s,
    and gained full cross-platform support in 2007.\punckern\endnote{Jonathan Kew,
    ``\XeTeX{} Live''\quotekern, \textit{TUGboat} 29, no.~1 (2007)}

\item[\LuaTeX] is similar to \XeTeX{} in its native Unicode and modern font support.
    It also embeds the Lua scripting language into the engine,
    exposing an interface for package and document authors.
    It first appeared in 2007 and is developed by a core team of
    Hans Hagen, Hartmut Henkel, Taco Hoekwater,
    and Luigi Scarso.\punckern\endnote{\http{www.luatex.org}}
\end{description}

Building \TeX{} today is an\dots{} interesting endeavor.
When it was written in the late 1970s,
there were no large, well-documented open-source projects for students to study,
so Knuth set out to make \TeX{} into one.
As part of this effort, \TeX{} was written in a style he calls
\introduce{literate programming}: opposite most programs---where
documentation is interspersed throughout the code---Knuth wrote \TeX{} as a book,
with the code interspersed between paragraphs.
This mix of English and code is called \texttt{WEB}.\punckern\footnote{Knuth
also released a pair of companion programs named
\texttt{TANGLE} and \texttt{WEAVE}.
The former extracts the book---as \TeX, of course---and the latter
produces \TeX's Pascal source code.}

Unsurprisingly, most modern systems don't have good tooling for the late 1970s
dialect of Pascal that \TeX{} was written in,
so present-day distributions use another program,
\texttt{web2c}, to convert its \texttt{WEB} source into C code.
pdf\TeX{} and \XeTeX{} are built by combining the result with other C
and \cpp{} sources.
Instead of following this complicated process,
the \LuaTeX{} authors hand-translated Knuth's Pascal into C.
They have used the resulting code since 2009.\punckern\endnote{%
Taco Hoekwater, \textit{\LuaTeX{} says goodbye to Pascal}
(MAPS 39, Euro\TeX{} 2009),
\https{www.tug.org/TUGboat/tb30-3/tb96hoekwater-pascal.pdf}}



\setlength\parskip{0.8\baselineskip}
\setlength\parindent{0pt}

\chapter{Additional Resources}
\label{resources}

\section{For \texorpdfstring{\LaTeX}{LaTeX}}

As promised at the start, this book is incomplete.
To keep things short,
major \LaTeX{} features---like figures, captions, and graphics---haven't
been discussed.
Use some of these resources to fill in the gaps:

\begin{adjustwidth}{1.5em}{0pt}
The \LaTeX{} Wikibook, at \url{https://en.wikibooks.org/wiki/LaTeX}

The \TeX{} Stack Exchange, at \url{https://tex.stackexchange.com/}

\textit{The Not So Short Introduction to \LaTeX}, \\
available at \url{https://www.ctan.org/tex-archive/info/lshort/english/}

The Share\LaTeX{} knowledge base, at \url{https://www.sharelatex.com/learn}

\end{adjustwidth}

\section{For typography}

We've spent most of our time here focusing \emph{what} you can do with \LaTeX,
and little on \emph{how} you should use it to create well-designed documents.
Read on:

\begin{adjustwidth}{1.5em}{0pt}
\textit{Practical Typography}, by Matthew Butterick. \\
Available (for free!) at \url{https://practicaltypography.com}

\textit{Stop Stealing Sheep \& Find Out How Type Works}, by Erik Spiekermann

\textit{Thinking With Type}, by Ellen Lupton

\textit{The Elements of Typographic Style}, by Robert Bringhurst

\textit{Detail in Typography}, by Jost Hochuli
\end{adjustwidth}

\backmatter

\setkomafont{chapter}{\Huge\itshape}

{\raggedright
\renewcommand\makeenmark{\theenmark.\enspace}
% Chicago Manual of Style, Notes & Bib style, ish.
% http://www.chicagomanualofstyle.org/tools_citationguide/citation-guide-1.html
\theendnotes
}

% Redefine cleardoublepage so the Colophon doesn't demand a front page.
% From https://tex.stackexchange.com/a/24068/92465
{\let\cleardoublepage\clearpage \chapter{Colophon}}

This guide was typeset with \LuaLaTeX{} in Garamond Premier by Robert Slimbach.
His revival is based on roman type by
\otford{16}{th} century French
punchcutter Claude Garamond.
Italics are inspired by the work of Garamond's contemporary Robert Granjon.

Monospaced items are set in Matthias Tellen's
\href{https://madmalik.github.io/mononoki/}{\texttt{mononoki}},
a typeface designed to work well on both low-resolution computer monitors
and in high-resolution print.

Captions are set in
\href{http://www.fontbureau.com/NHG/}{\textsf{\small Neue Haas Grotesk}},
a Helvetica restoration by Christian Schwartz.
Other digitizations of the classic Swiss typeface are based on fonts made for
Linotype and phototypesetting machines,
resulting in digital versions with all the compromises and kludges from those
past two generations of printing technology.
Schwartz based his work on Helvetica's original drawings,
producing a design faithful to the original cold metal type.

{\fontspec[Ligatures=TeX, Scale=MatchLowercase]{Futura-Boo}URW Futura}
makes a few guest appearances.
Originally released in 1927 by Paul Renner,
Futura has found itself almost everywhere,
from advertising and political campaigns to the moon.
Douglas Thomas's recent history of the typeface,
\textit{Never Use Futura}, is a fantastic read.

Various bits of non-Latin text are set in
{\fontspec[Ligatures=TeX,Scale=MatchLowercase]{NotoSerif-Regular}Noto},
a type family by Google that covers \emph{every} language
in the Unicode standard.

Finally,
{\lm Latin Modern}---the OpenType version of Knuth's Computer Modern used throughout
the book---as well
as {\fontspec[Scale=MatchUppercase]{TeX Gyre Termes}\TeX{} Gyre Termes}---the
free alternative to Times Roman seen on page \pageref{typography}---are from
the digital type foundry of Grupa Użytkowników Systemu \TeX{},
the Polish \TeX{} Users' Group.
An overview of their excellent work can be found at the following locations:\\
\url{http://www.gust.org.pl/projects/e-foundry/latin-modern} \\
\url{http://www.gust.org.pl/projects/e-foundry/tex-gyre}.


\end{document}
