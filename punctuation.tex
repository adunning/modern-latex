\chapter{Punctuation}
\label{punctuation}

You would much rather encounter a panda that eats
shoots and leaves than one that eats, shoots,
and leaves.\punckern\endnote{Lynne Truss,
\textit{Eats, Shoots \& Leaves} (New York, 2003)}
Punctuation is a vital part of writing,
and there's more to it than your keyboard suggests.

\section{Quotation marks}

\LaTeX{} doesn't automatically convert ``straight'' quotes
into correctly-facing ``curly'' ones:
\begin{leftfigure}
\begin{lstlisting}
"This isn't right."
\end{lstlisting}
\end{leftfigure}
will get you
\begin{leftfigure}
\lm%
"This isn't right."
\end{leftfigure}
Instead, use \texttt{`} for opening quotes and \texttt{'} for closing
quotes.\punckern\footnote{Don't use \texttt{"} for closing double quotes.
Not only does \texttt{``example"} look a bit unbalanced,
but \texttt{"} is used as a formatting command when typesetting certain
languages, like German. (See \chapref{i18n} for more on international
typesetting.)}
\begin{leftfigure}
\begin{lstlisting}
``It depends on what the meaning of the word `is' is.''
\end{lstlisting}
\end{leftfigure}
quotes a former \acronym{us} president as,
\begin{leftfigure}
\lm%
``It depends on what the meaning of the word `is' is.''
\end{leftfigure}

\section{Hyphens and dahes}

Though they look similar,
hyphens (\,-\,), en dashes (\,--\,),
em dashes (\,---\,), and minus signs (\,$-$\,)
serve different purposes:
\begin{description}
\item[Hyphens] have a few uses:\endnote{Butterick,
    ``Hyphens and dashes''\quotekern,
    \textit{Practical Typography},
    \url{https://practicaltypography.com/hyphens-and-dashes.html}}
    \begin{itemize}[leftmargin=*]
    \item They allow a word to be split between the end of one line and the
        start of the next.
        \LaTeX{} handles this automatically,
        so you don't have to worry about manually hyphenating your lines.
    \item Some compound words use hyphens, like \emph{long-range}
        and \emph{field-effect}.
    \item They are used in phrasal adjectives.
        If I ask for ``five dollar bills''\punckern,
        do I want five \$1 bills, or several \$5 bills?
        It's clearer that I mean the latter when set as
        \emph{five-dollar bills}.
    \end{itemize}
    In \LaTeX{}, they are produced with the hyphen key (\,\texttt{-}\,).

\item[En dashes] indicate ranges such as ``pages 4--12''\quotekern,
    or separations like the ``US--Canada border''\quotekern.
    They are set with two adjacent hyphens (\,\texttt{--}\,).

\item[Em dashes] can be used to separate clauses of a sentence.
    Other punctuation marks---like parenthesis or commas---play a similar
    role.
    They are set with three adjacent hyphens (\,\texttt{---}\,).

\item[Minus signs] are used exclusively for negative quantities and
    mathematical expressions.
    They are often similar in length to an en dash,
    but sit at a different height.
    They are set with the hyphen key when in a math environment
    (see \chapref{math}), or with \verb|\textminus|.
\end{description}

\section{Ellipses}

The three dots used to indicate a pause or omission are called
\introduce{ellipses}.
They are set using \verb|\ldots|.
\begin{leftfigure}
\begin{lstlisting}
I'm\ldots{} not sure.
\end{lstlisting}
\end{leftfigure}
becomes
\begin{leftfigure}
\lm%
I'm\ldots{} not sure.
\end{leftfigure}

\section{Spacing}

As we discovered in our first example,
\LaTeX{} automatically inserts extra space between periods and whatever
follows them---presumably the start of the next sentence.
Sometimes, this isn't what we want!
Consider honorifics like Mr.\ and Ms., for example.
In situations like these, we also want to keep \LaTeX{} from starting a new line
after the period.
This calls for a \introduce{non-breaking space}, which we set with a tilde.
\begin{leftfigure}
\begin{lstlisting}
Please call Ms.~Shrdlu.
\end{lstlisting}
\end{leftfigure}
produces proper spacing:
\begin{leftfigure}
\lm%
Please call Ms.~Shrdlu.
\end{leftfigure}

In other occasions, such as when we abbreviate units of
measurement,\punckern\footnote{There are also dedicated packages for doing so,
like \texttt{siunitx}.}
we want spaces that are thinner than usual inter-word ones.
For these, we use \verb|\,|:
\begin{leftfigure}
\begin{lstlisting}
Launch in 2\,h 10\,m.
\end{lstlisting}
\end{leftfigure}
announces
\begin{leftfigure}
\lm%
Launch in 2\,h 10\,m.
\end{leftfigure}
Thin spaces are also useful when separating single and double quotation marks.
\begin{leftfigure}
\begin{lstlisting}
``She exclaimed, `I can't believe it!'\,''
\end{lstlisting}
\end{leftfigure}
gives us the quote:
\begin{leftfigure}
\lm%
``She exclaimed, `I can't believe it!'\,''
\end{leftfigure}

\section{What next?}
\begin{itemize}
\item Learn other commands for spacing, such as \verb|\:|, \verb|\;|,
    \verb|\enspace|, and \verb|\quad|.
\item Discover the typographical origins of terms like \introduce{en},
    \introduce{em}, and \introduce{quad}.
\item Familiarize yourself with the difference between \texttt{/} and
    \verb|\slash|.
\end{itemize}
