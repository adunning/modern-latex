\chapter{Installation}

You install \LaTeX{} on your computer by downloading a \introduce{distribution},
which comes with:
\begin{enumerate}
\item \LaTeX, the program---the thing that turns text files into nicely-typeset
    documents.\footnote{Well, \LaTeX{} the program\textbf{s},
but we're getting to that.}
\item A common set of \LaTeX{} \introduce{packages}.
    Packages are pieces of \LaTeX{} code that you can place in your document to
    do all sorts of things, like add new commands or change its style.
    We'll see lots of them in action throughout the book.
\item Miscellaneous tools, such as editors and the like.
\end{enumerate}
Each major operating system has its own \LaTeX{} distribution:
\begin{description}
\item[Mac OS] has Mac\TeX. Grab it from \url{http://www.tug.org/mactex}
    and follow the instructions there to install it.

\item[Windows] has Mik\TeX.
    Install it from \url{https://miktex.org/download}
    as described by the tutorial.
    Unique to the other \LaTeX{} distributions,
    Mik\TeX{} offers the ability to automatically find and download additional
    packages if your document uses one that isn't currently on your computer.

\item[Linux and BSD] usually use \TeX{} Live.
    Like most software, this is usually provided through your
    \acronym{os}'s package manager (e.g., Apt, RPM, Pacman, Zypper, etc.).
    Use yours to install \monobox{texlive-core},
    \monobox{texlive-luatex} and \monobox{texlive-xetex}.\punckern\footnote{%
    Note that some distributions (like Arch) package all three into a
    single \monobox{texlive-core} package.}
    As you continue to use \LaTeX, you may start to need less-common packages,
    which are usually found in distro packages with names like
    \texttt{texlive-latexextra}, \texttt{texlive-science},
    and so on.\punckern\footnote{Of
    course, package names, and how \LaTeX{} packages are split between
    distro packages, may vary from one Linux distribution to another.
    These names are some of the most common.}
\end{description}

\section{Editors}

\LaTeX{} source files are regular text files,
so you're free to create them with the usual choices of Vim, Emacs, Gedit,
Sublime, Notepad\plusplus, and so on.\punckern\footnote{If you have never used
any of these, you owe it to yourself to try a few.
They're very popular among programmers and other folks who shuffle text around
screens all day. Just don't use Notepad. Life is too short.}
However, there are also editors designed specifically for \LaTeX{},
which often come with their own built-in \acronym{pdf} viewer.
You can find a fairly comprehensive list on the \LaTeX{} Wikibook,
in its installation chapter. (See Appendix~\ref{resources}.)
