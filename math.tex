\chapter{Mathematics}

\LaTeX{} is excellent at typesetting mathematics, both inline with body text,
e.g., $x_n^2+y_n^2=r^2$, and as standalone formulas:
\[\sum_{n=0}^{\infty} \frac{f^{(n)} (a)}{n!} (x - a)^n\]
The former is created with \verb|$...$| or \verb|\(...\)|,
and the latter with \verb|\[...\]|.
Inside these math environments, the rules of \LaTeX{} change:
\begin{itemize}
\item Most spaces and line breaks are ignored completely,
    and the engine usually makes spacing decisions for you based on
    typographical conventions for mathematics.
    \verb|$x+y+z$| and \verb|$x + y + z$| both give you $x+y+z$.
\item Empty lines are not allowed---each formula must occupy a single
    ``paragraph''\quotekern.
\item Letters are automatically italicized, as they are assumed to be variables.
\end{itemize}
To return to normal ``text mode'' inside a formula, use the \verb|\text| command.
Other formatting commands mentioned in \chapref{formatting} work as well.
From
\begin{leftfigure}
\begin{lstlisting}
\[\text{fake formulas} = \textbf{annoyed mathematicians}\]
\end{lstlisting}
\end{leftfigure}
we get
\[\text{\lm fake formulas} = \textbf{\lm annoyed mathematicians}\]

\section{Examples}

Typesetting mathematics is arguably the raison d'être of
\LaTeX,\punckern\footnote{Well, \TeX} but the topic is so broad that giving
it decent coverage would take up half the book.
Given how wide the field of mathematics is,
there are \emph{many} different commands and environments.
Here, more so than any other topic,
you owe it to yourself to find some real references and learn what the system
is capable of.
Before moving on, though, let's show some examples of what \LaTeX{}
can do.
\newpage

\begin{enumerate}
\item \verb|x = \frac{-b \pm \sqrt{b^2 - 4 a c}}{2a}|
    \[x = \frac{-b \pm \sqrt{b^2 - 4 a c}}{2a} \]

\item \verb|e^{j \theta} = \cos(\theta) + j \sin(\theta)|
    \[e^{j \theta} = \cos(\theta) + j \sin(\theta)\]

\item
\begin{verbatim}
\begin{bmatrix}
x' \\
y'
\end{bmatrix} =
\begin{bmatrix}
\cos \theta &  -\sin\theta \\
\sin \theta & \cos \theta
\end{bmatrix}
\begin{bmatrix}
x \\
y
\end{bmatrix}
\end{verbatim}
\[
\begin{bmatrix}
x' \\
y'
\end{bmatrix} =
\begin{bmatrix}
\cos \theta &  -\sin\theta \\
\sin \theta & \cos \theta
\end{bmatrix}
\begin{bmatrix}
x \\
y
\end{bmatrix}
\]

\item
\begin{verbatim}
\oint_{\partial \Sigma} \mathbf{E} \cdot \mathrm{d}\boldsymbol{\ell}
    = - \frac{\mathrm{d}}{\mathrm{d}t}
      \iint_{\Sigma} \mathbf{B} \cdot \mathrm{d}\mathbf{S}
\end{verbatim}
    \[\oint_{\partial \Sigma} \mathbf{E} \cdot \mathrm{d}\boldsymbol{\ell}  = - \frac{\mathrm{d}}{\mathrm{d}t} \iint_{\Sigma} \mathbf{B} \cdot \mathrm{d}\mathbf{S}\]
\end{enumerate}
