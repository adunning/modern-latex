\chapter{Mathematics}
\label{math}

\LaTeX{} excels at typesetting mathematics, both in body text,
e.g., $x_n^2+y_n^2=r^2$, and as standalone formulas:
\[\sum_{n=0}^{\infty} \frac{f^{(n)} (a)}{n!} (x - a)^n\]
The former is typed inside \verb|$...$| or \verb|\(...\)|,
and the latter within \verb|\[...\]|.
In these math environments, the rules of \LaTeX{} change:
\begin{itemize}
\item Most spaces and line breaks are ignored,
    and spacing decisions are made for you based on
    typographical conventions for mathematics.
    \verb|$x+y+z$| and \verb|$x + y + z$| both give you $x+y+z$.
\item Empty lines are not allowed---each formula occupies a single
    ``paragraph''\quotekern.
\item Letters are automatically italicized, as they are assumed to be variables.
\end{itemize}
To return to normal ``text mode'' inside a formula, use the \verb|\text| command.
Other formatting commands mentioned in \chapref{formatting} work as well.
From
\begin{leftfigure}
\begin{lstlisting}
\[ \text{fake formulas} = \textbf{annoyed mathematicians} \]
\end{lstlisting}
\end{leftfigure}
we get
\[ \text{\lm fake formulas} = \textbf{\lm annoyed mathematicians} \]

\section{Examples}

Typesetting mathematics is arguably the raison d'être of
\LaTeX,\punckern\footnote{Well, \TeX} but the topic is so broad that giving
it decent coverage would take up half of this book.
Given how wide the field of mathematics is,
there are \emph{many} different commands and environments.
Here, more so than any other topic,
you owe it to yourself to find some real references and learn what \LaTeX{}
is capable of.
Before moving on, though, let's show some examples of what it can do.
\newpage

\begin{enumerate}
% We're going to cheat a bit here,
% since getting \sin, \cos, etc. to be Latin Modern is oddly tricky,
% even when using the unicode-math package
\item \verb|x = \frac{-b \pm \sqrt{b^2 - 4 a c}}{2a}|
    \[x = \frac{-b \pm \sqrt{b^2 - 4 a c}}{2a} \]

\item \verb|e^{j \theta} = \cos(\theta) + j \sin(\theta)|
    \[e^{j \theta} = \cos(\theta) + j \sin(\theta)\]

\item
\begin{verbatim}
\begin{bmatrix}
x' \\
y'
\end{bmatrix} =
\begin{bmatrix}
\cos \theta &  -\sin\theta \\
\sin \theta & \cos \theta
\end{bmatrix}
\begin{bmatrix}
x \\
y
\end{bmatrix}
\end{verbatim}
\[
\begin{bmatrix}
x' \\
y'
\end{bmatrix} =
\begin{bmatrix}
\cos \theta &  -\sin\theta \\
\sin \theta & \cos \theta
\end{bmatrix}
\begin{bmatrix}
x \\
y
\end{bmatrix}
\]

\item
\begin{verbatim}
\oint_{\partial \Sigma} \mathbf{E} \cdot
\mathrm{d}\boldsymbol{\ell}
= - \frac{\mathrm{d}}{\mathrm{d}t}
    \iint_{\Sigma} \mathbf{B} \cdot \mathrm{d}\mathbf{S}
\end{verbatim}
    \[\oint_{\partial \Sigma} \mathbf{E} \cdot \mathrm{d}\boldsymbol{\ell}  = - \frac{\mathrm{d}}{\mathrm{d}t} \iint_{\Sigma} \mathbf{B} \cdot \mathrm{d}\mathbf{S}\]
\end{enumerate}
