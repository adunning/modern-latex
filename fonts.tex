\chapter{Fonts}
\label{fonts}

Digital type has changed almost entirely since \LaTeX{} was released in the
1980s.
Originally, it used \MF, a system designed by Donald Knuth specifically
for \TeX{}.
As time went on, support for PostScript\footnote{One of
Adobe's original claims to fame,
PostScript is a language for defining and drawing computer graphics,
including type. It remains in use today.} fonts was added.
Today, \LuaLaTeX{} and \XeLaTeX{} offer support for the two formats you're
likely to encounter on your computer:
TrueType and OpenType.

\begin{description}
\item[TrueType] was developed by Apple and Microsoft in the late 1980s.
    Most of the fonts that come pre-installed on your system are likely
    in this format.
    TrueType files generally end in a \monobox{.ttf} extension.
\item[OpenType] was first released in 1996 by Microsoft and Adobe.
    One of its major improvements over TrueType is the ability to embed
    multiple versions of glyphs in a single font file.
    We'll see this feature in action through much of this chapter.
    OpenType files generally end in an \monobox{.otf} extension.
\end{description}

\section{Changing typefaces}

By default, \LuaLaTeX{} and \XeLaTeX{} use Latin Modern,
an OpenType rendition of \LaTeX's original type family, Computer Modern.
While Latin Modern is a high-quality family of typefaces,
you may want to use others for your document.
This is done through the \texttt{fontspec} package:
\begin{leftfigure}
\begin{lstlisting}
\documentclass{article}

\usepackage{fontspec}
\setmainfont[Ligatures=TeX]{Source Serif Pro}
\setsansfont[Ligatures=TeX]{Source Sans Pro}
\setmonofont{Source Code Pro}

\begin{document}
Hello, Source type family! Neat---no? \\
\sffamily Let's try sans serif! \\
\ttfamily Let's try monospaced!
\end{document}
\end{lstlisting}
\end{leftfigure}
should produce something like:\footnote{Assuming, of course,
that you have Adobe's open-source fonts installed.\punckern\endnote{Adobe's
open-source type can be found freely at \url{https://github.com/adobe-fonts}.}}
\begin{leftfigure}
\fontspec[Ligatures=TeX]{Source Serif Pro} Hello, Source type family! Neat---no? \\
\fontspec[Ligatures=TeX]{Source Sans Pro} Let's try sans serif! \\
\fontspec{Source Code Pro} Let's try monospaced!
\end{leftfigure}
The \verb|Ligatures=TeX| option allows you to use the standard ligatures
mentioned in \chapref{punctuation} instead of characters that are
unlikely to be on your keyboard.
% No need repeating ourselves here?
%For example, \verb|---| can be used to create em dashes (—),
%quotes can be typed \verb|``like this''| instead of \verb|“like this”|,
%and so on.
You probably don't want this for your monospaced type, however,
since things set in it---such as code---are usually meant to be displayed
verbatim. You don't want, for example, \verb|"Hello!"| to turn into
\verb|“Hello!“|

\section{Selecting font files}

A typical typeface might come packaged as four files to represent its
weights and styles:
normal,
\textit{italics},
\textbf{bold}, and
\textit{\textbf{bold italics}}.
Given a typefaces's name,
\texttt{fontspec} can generally deduce the appropriate file
names.\punckern\footnote{This is one of the places \XeLaTeX{} and \LuaLaTeX{}
differ in a way that's noticeable to the casual user.
The former gernally uses system libraries---such as FontConfig on Linux---to
locate files, given a typeface's name.
The latter has its own font loader,
based on code from FontForge.\punckern\endnote{\textit{\LuaTeX{} Reference}
(February 2017, Version 1.0.4), 10}
The expected name of a font might differ between the two engines---refer
to the \texttt{fontspec} manual for details.}
However, many typefaces come in more than two weights---some versions of Futura,
for example, come in
{\fontspec[Scale=MatchLowercase]{Futura-Lig}light},
{\fontspec[Scale=MatchLowercase]{Futura-Boo}book},
{\fontspec[Scale=MatchLowercase]{Futura-Med}medium},
{\fontspec[Scale=MatchLowercase]{Futura-Dem}demi},
{\fontspec[Scale=MatchLowercase]{Futura-Bol}bold}, and
{\fontspec[Scale=MatchLowercase]{Futura-ExtBol}extra bold}.
Sometimes
{\fontspec[Scale=MatchLowercase]{FuturaSc-Boo}\textsc{small caps}}
are stored as separate files as well.

We might want to hand-pick weights to achieve a certain look or better match the
weights of other fonts in our document.\punckern\footnote{Consider how much
better the {\fontspec[Scale=MatchLowercase]{Futura-Boo}book weight} of Futura
blends in with the surrounding text compared to
{\fontspec[Scale=MatchLowercase]{Futura-Lig}light}
or
{\fontspec[Scale=MatchLowercase]{Futura-Med}medium}.}
Continuing to use Futura as an example,
say we want to use the ``book'' weight for our default weight,
``demi'' for bold,
and the font files are named:
\begin{itemize}
\item \monobox{Futura-Boo} for our
    {\fontspec[Scale=MatchLowercase]{Futura-Boo}upright book weight}
\item \monobox{Futura-BooObl} for our
    {\fontspec[Scale=MatchLowercase]{Futura-BooObl}oblique book weight}
\item \monobox{FuturaSC-Boo} for
    {\fontspec[Scale=MatchLowercase]{FuturaSC-Boo}small caps, book weight}
\item \monobox{Futura-Dem} for
    {\fontspec[Scale=MatchLowercase]{Futura-Dem}upright demi(bold)}
\item \monobox{Futura-DemObl} for
    {\fontspec[Scale=MatchLowercase]{Futura-DemObl}oblique demibold}
\end{itemize}

Our font setup might resemble:
\begin{leftfigure}
\begin{lstlisting}
\usepackage{fontspec}
\setmainfont[
    Ligatures=TeX,
    UprightFont = *-Boo,
    ItalicFont = *-BooObl,
    SmallCapsFont = *SC-Boo,
    BoldFont = *-Dem,
    BoldItalicFont = *-DemObl
]{Futura}
\end{lstlisting}
\end{leftfigure}
Note that instead of typing out \monobox{Futura-Boo},
\monobox{Futura-BooObl}, and so on, we can use \texttt{*} to insert the base name.

\section{Scaling}

Using different typefaces to create a cohesive experience is tricky,
especially since---as mentioned in \chapref{formatting}---different typefaces
might look completely different at the same point size.
\texttt{fontspec} can help a bit here by scaling fonts to match either the
x-height or the cap height of your main font with
\verb|Scale=MatchLowercase| or \verb|Scale=MatchUppercase|,
respectively.\footnote{A useful way to sidestep this issue is to user fewer
typefaces in your design. Even just one or two typefaces,
with sufficient features, can produce amazing results.}


\section{OpenType features}

As mentioned above, one of OpenType's defining features is the ability to store
multiple variations of a typeface's glyphs in a single file and allow users
to switch between them.
All of these features can be specified as an option to \verb|\setmainfont|
and friends, or enabled for the current group with
\verb|\addfontfeature|.

\subsection{Ligatures}

Many typefaces use \introduce{ligatures}, where multiple characters are combined
into a single glyph.\punckern\footnote{Trivia: glyphs such as the ampersand
(\,\&\,) and the German Eszett (\,ß\,) evolved
from ligatures.}
While the practice has historical roots,\punckern\footnote{Ligatures fell out
of style somewhat during the 20{\addfontfeature{VerticalPosition=Superior}th}
century due to limitations of printing technology---including early
computer typesetting---and the increased popularity of sans serif typefaces,
which often lack them.
Today they are making a comeback,
thanks in no small part to their support in OpenType.}
OpenType groups ligatures into three categories:
\begin{description}
\item[Standard] TODO
fi, ff, ffi, fl, and ffl.
\item[Discretionary] TODO
\item[Historic] TODO
\end{description}

Mention \texttt{selnolig}!

\subsection{Figures}

Lining, oldstyle, tabular, proportional, oh my!

\subsection{Superiors and inferiors}

Not about a trip to HR

\subsection{Stylistic alternates}
