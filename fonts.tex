\chapter{Fonts}
\label{fonts}

Digital fonts have changed almost entirely in the past thirty years.
Originally, \LaTeX{} used \MF,
a system designed by Donald Knuth specifically for \TeX{}.
As time went on, support for PostScript\footnote{One of
Adobe's original claims to fame,
PostScript is a language for defining and drawing computer graphics,
including type. It remains in use today.} fonts was added.
Today, \LuaLaTeX{} and \XeLaTeX{} support the formats you're
most likely to encounter on your computer:
TrueType and OpenType.\punckern\footnote{Mac versions of \LaTeX{} also support
Apple's \acronym{aat}, but let's limit this discussion to
more ubiquitous formats.}

\begin{description}
\item[TrueType] was developed by Apple and Microsoft in the late 1980s.
    Most of the fonts that come pre-installed on your system are likely
    in this format.
    TrueType files generally end in a \monobox{.ttf} extension.
\item[OpenType] was first released by Microsoft and Adobe in 1996.
    Improvements over TrueType include its ability to embed
    various ``features''\quotekern, such as alternative glyphs
    and spacing options, into a single font file.
    OpenType files usually end in an \monobox{.otf} extension.
\end{description}

\section{Changing fonts}

By default, \LuaLaTeX{} and \XeLaTeX{} use Latin Modern,
an OpenType rendition of \LaTeX's original type family, Computer Modern.
While these are high-quality fonts,
they're probably not the only ones you ever want to use.
For others, we turn to the \texttt{fontspec} package:
\begin{leftfigure}
\begin{lstlisting}
\documentclass{article}

\usepackage{fontspec}
\setmainfont[Ligatures=TeX]{Source Serif Pro}
\setsansfont[Ligatures=TeX]{Source Sans Pro}
\setmonofont{Source Code Pro}

\begin{document}
Hello, Source type family! Neat---no? \\
\sffamily Let's try sans serif! \\
\ttfamily Let's try monospaced!
\end{document}
\end{lstlisting}
\end{leftfigure}
should produce something like\footnote{Assuming, of course,
that you have Adobe's open-source fonts installed.\punckern\endnote{Adobe's
open-source typefaces are freely available at
\https{github.com/adobe-fonts}}}
\begin{leftfigure}
\fontspec[Ligatures=TeX]{Source Serif Pro} Hello, Source type family! Neat---no? \\
\fontspec[Ligatures=TeX]{Source Sans Pro} Let's try sans serif! \\
\fontspec{Source Code Pro} Let's try monospaced!
\end{leftfigure}
The \verb|Ligatures=TeX| option enables the standard punctuation
shortcuts discussed in \chapref{punctuation} (e.g., creating en dashes with
\texttt{--} or curly quotes with \texttt{``}) instead of forcing you
to type the the corresponding characters, which probably aren't on your keyboard.
You often don't want these for monospaced type, though,
since the text you set in it---such as code---is meant to be displayed
verbatim. \verb|"Hello!"| shouldn't turn into
\verb|“Hello!“|.

\section{Selecting font files}

Typefaces come packaged as multiple files for their various
weights and styles---a typical set includes upright,
\textit{italics},
\textbf{bold}, and
\textit{\textbf{bold italics}}.
Given a typeface's name,
\texttt{fontspec} can usually deduce the names of its
files.\punckern\footnote{This is
one of the places where \XeLaTeX{} and \LuaLaTeX{}
differ in a way that's noticeable to the casual user.
The former generally uses system libraries---such as FontConfig on Linux---to
locate files from a typeface's name.
The latter has its own font loader,
based on code from FontForge.\punckern\endnote{\textit{\LuaTeX{} Reference}
(Version 1.0.4, February 2017), 10}
The expected name of a font might differ between the two engines---refer
to the \texttt{fontspec} manual for details.
If you have trouble getting \LaTeX{} to find your fonts,
try rebuilding your system's font cache.
(Figuring out how to do so is left as a web search for the reader.)}

However, many typefaces come in more than two weights---some cuts of Futura,
for example, come in
{\fontspec[Scale=MatchLowercase]{Futura-Lig}light},
{\fontspec[Scale=MatchLowercase]{Futura-Boo}book},
{\fontspec[Scale=MatchLowercase]{Futura-Med}medium},
{\fontspec[Scale=MatchLowercase]{Futura-Dem}demi},
{\fontspec[Scale=MatchLowercase]{Futura-Bol}bold}, and
{\fontspec[Scale=MatchLowercase]{Futura-ExtBol}extra bold}.
Sometimes
{\fontspec[Scale=MatchLowercase]{FuturaSc-Boo}\textsc{small caps}}
are stored in separate files as well.\punckern\footnote{OpenType allows
small caps to be placed in the same file(s) as the other glyphs for a given
weight.
If your font supports this, you don't need to do anything---\texttt{fontspec}
will dutifully switch to them whenever you use
\monobox{\textbackslash textsc} or \monobox{\textbackslash scshape}.
But for TrueType, and for OpenType fonts that don't take advantage of this
feature, you'll have to load a separate file as shown here.}
We might want to hand-pick weights to achieve a certain look or better match the
weights of other fonts in our document.\punckern\footnote{Compare how
{\fontspec[Scale=MatchLowercase]{Futura-Lig}the light,}
{\fontspec[Scale=MatchLowercase]{Futura-Boo}book,}
{\fontspec[Scale=MatchLowercase]{Futura-Med}and medium weights}
of Futura look compared to surrounding type on this page.}
Continuing to use Futura as an example,
say we want to use the ``book'' weight as our default
and ``demi'' for bold.
Assuming the font files are named:
\begin{itemize}
\item \monobox{Futura-Boo} for our
    {\fontspec[Scale=MatchLowercase]{Futura-Boo}upright book weight}
\item \monobox{Futura-BooObl} for our
    {\fontspec[Scale=MatchLowercase]{Futura-BooObl}oblique book weight}
\item \monobox{FuturaSC-Boo} for
    {\fontspec[Scale=MatchLowercase]{FuturaSC-Boo}small caps, book weight}
\item \monobox{Futura-Dem} for
    {\fontspec[Scale=MatchLowercase]{Futura-Dem}upright demi(bold)}
\item \monobox{Futura-DemObl} for
    {\fontspec[Scale=MatchLowercase]{Futura-DemObl}oblique demibold}
\end{itemize}

\noindent Our setup might resemble:
\begin{leftfigure}
\begin{lstlisting}
\usepackage{fontspec}
\setmainfont[
    Ligatures=TeX,
    UprightFont = *-Boo,
    ItalicFont = *-BooObl,
    SmallCapsFont = *SC-Boo,
    BoldFont = *-Dem,
    BoldItalicFont = *-DemObl
]{Futura}
\end{lstlisting}
\end{leftfigure}
Note that instead of typing out \monobox{Futura-Boo},
\monobox{Futura-BooObl}, and so on, we can use \texttt{*} to insert the base name.

\section{Scaling}

Creating a cohesive design with multiple fonts is tricky,
especially since typefaces might look completely different
at the same point size.
\texttt{fontspec} can help a bit here by scaling fonts to match either the
x-height or the cap height of your main font with
\verb|Scale=MatchLowercase| or \verb|Scale=MatchUppercase|,
respectively.\footnote{One way to sidestep this issue is to have fewer
typefaces in your design. Even just one or two,
used carefully, can produce amazing results.}


\section{OpenType features}

As mentioned at the start of the chapter,
OpenType fonts provide various features that can be turned on and off.
In \LaTeX{}, these are controlled through \texttt{fontspec}
with optional arguments to \verb|\setmainfont| and friends.
They can also be set for the current group with
\verb|\addfontfeature|.
Let's examine a few common ones.

\subsection{Ligatures}

Many typefaces use \introduce{ligatures}, which combine multiple characters
into a single glyph.\punckern\footnote{Ligatures fell out
of style during the 20{\addfontfeature{VerticalPosition=Superior}th}
century due to limitations of printing technology and the increased popularity
of sans serif typefaces, which often lack them.
Today they are making a comeback,
thanks in no small part to their support in OpenType.}
OpenType groups ligatures into three categories:
\begin{description}
\item[Standard] ligatures remedy spacing problems a typeface might otherwise have.
    Consider the lowercase letters f and i: in many serif typefaces,
    these combine to form the ligature fi,
    which avoids awkward spacing between f's ascender and i's dot
    {\addfontfeature{Ligatures=CommonOff} (\,fi\,)}.
    Other common examples in English writing include ff,
    ffi, fl, and ffl.
    Standard ligatures are enabled by default.
\item[Discretionary] ligatures, such as
    {\addfontfeature{Ligatures=Discretionary}ct},
    are offered by some fonts.
    They are disabled by default
    but can be enabled with
    \verb|Ligatures=Discretionary|.
\item[Historical] ligatures are ones which have fallen out of common use,
    such as those with a \introduce{long~s} (e.g., ſt).
    These are also disabled by default
    but can be enabled with \verb|Ligatures=Historic|.
\end{description}
Multiple options can be grouped together.
Say you want discretionary ligatures.
In the likely event that you also want \verb|Ligatures=TeX|,
you would enable both with
\verb|Ligatures={TeX,Discretionary}|.
Ligatures can also be disabled with corresponding \verb|*Off|
options. If you want to stop using discretionary ligatures for some passage,
\begin{leftfigure}
\begin{lstlisting}
{\addfontfeature{Ligatures=DiscretionaryOff}...}
\end{lstlisting}
\end{leftfigure}
does the trick.

Some words are arguably typeset better without ligatures---a classic example
is shelfful.\punckern\endnote{Knuth, \textit{The \TeX book},
(Addison-Wesley, 1986), 19}
You can either manually prevent ligatures with a zero-width
space, e.g., \verb|shelf\hspace{0pt}ful|,
or use the \texttt{selnolig} package to handle most of these cases automatically.

\subsection{Figures}

When setting figures,\punckern\footnote{\introduce{Figure}
here refers to what some might call a \introduce{numeral} or
\introduce{digit}---i.e., 0, 1, 2, 3, 4, 5, 6, 7, 8, 9.
Typographers generally prefer the first term over the other two.}
you have two
choices to make: lining versus oldstyle,
and proportional versus tabular.
\introduce{Lining} figures, sometimes called \introduce{titling} figures,
have similar heights as capital letters:
\begin{leftfigure}
\addfontfeature{Numbers=LowercaseOff}
A B C D 1 2 3 4
\end{leftfigure}
\introduce{Oldstyle}, or \introduce{text} figures,
share more similarities with lowercase letters:
\begin{leftfigure}
Sitting cross-legged on the floor\dots{} 25 or 6 to 4?
\end{leftfigure}
For body text, either choice is fine, but oldstyle figures shouldn't
be combined with capital letters.
``F-15C'' looks odd, as does ``V2.3 Release''\quotekern.

{\addfontfeature{Numbers=LowercaseOff}
The terms \introduce{proportional} and \introduce{tabular} refer to spacing.
Tabular figures are set with a uniform width, such that 1 takes up
the same space as 8.
As their name suggests, this is great for tables and other scenarios
where figures must line up in columns:}
\begin{leftfigure}
\addfontfeature{Numbers={Tabular,LowercaseOff}}
\begin{tabular}{l|c r}
Item & Qty. & Price \\
\hline
Gadgets & 42 & \$5.37 \\
Widgets & 18 & \$12.76 \\
\end{tabular}
\end{leftfigure}
Proportional figures are the opposite---their spacing is, well\dots{}
\emph{proportional} to the width of the figure.
They are usually preferred in body text, where 1837
looks a bit nicer than
{\addfontfeature{Numbers=Tabular}1837}.

You select figures with the following options:
\begin{leftfigure}
\begin{tabular}{l l}
\texttt{Numbers=} & \texttt{Lining / Uppercase} \\
 & \texttt{OldStyle / Lowercase} \\
 & \texttt{Proportional} \\
 & \texttt{Tabular / Monospaced}
\end{tabular}
\end{leftfigure}
As with ligature options, these can be combined:
proportional lining figures are set
with \texttt{Numbers=\allowbreak\{Proportional,\allowbreak Lining\}},
and tabular oldstyle figures are set with
\texttt{Numbers=\allowbreak\{Tabular,\allowbreak OldStyle\}}.
Each option also has a corresponding \verb|*Off|
variant.\punckern\footnote{This is especially useful since fonts
select figures in different ways.
In some, \texttt{Numbers=\allowbreak Lining} doesn't work,
so you enable oldstyle figures with \monobox{Numbers=OldStyle}
and return to lining figures (the defaults) with
\texttt{Numbers=\allowbreak OldStyleOff}.}

Finally, some fonts provide \introduce{superior and inferior} figures,
which are used to set ordinals
(\otford{1}{st}, \otford{2}{nd} \otford{3}{rd}, \dots),
fractions (\,\otffrac{25}{624}\,), and so on.
They have the same weight as the rest of the font's characters,
offering a more consistent look than shrunken versions of full-sized figures.
(Compare the examples above to
{\fontspec[OpticalSize=0]{garamondpremrpro}%
\mbox{1\textsuperscript{st}},
\mbox{2\textsuperscript{nd}},
\mbox{3\textsuperscript{rd}},
and
\,\mbox{\textsuperscript{25}^^^^2044\textsubscript{624}}%
\,}.
Notice how this second set is too light compared to the surrounding
type.)
Superior figures are typeset with
\texttt{VerticalPosition=\allowbreak Superior},
and inferiors are set with \texttt{VerticalPosition=\allowbreak Inferior}.

\section{What next?}
\begin{itemize}
\item Learn how \texttt{fontspec} can choose between optical sizes based on
    point size, either automatically from ranges embedded in OpenType fonts,
    or manually using \texttt{SizeFeatures}.
\item Experiment with letter spacing---or \introduce{tracking}---with
    the \texttt{LetterSpace} option.
    Extra tracking is unnecessary in most cases,
    but can be useful to make \textsc{small caps}
    a bit more \acronym{readable}.
\end{itemize}
