\setlength\parskip{0.75\baselineskip}
\setlength\parindent{0pt}

\chapter{Additional Resources}
\label{resources}

\section{For \texorpdfstring{\LaTeX}{LaTeX}}

As promised at the start, this book is incomplete.
To keep it short,
major \LaTeX{} features---like figures, captions, tables, graphics,
and bibliographies---haven't been discussed.
Use some of these resources to fill in the gaps:
\begin{leftfigure}
The \LaTeX{} Wikibook, at \https{en.wikibooks.org/wiki/LaTeX}

\textit{The Not So Short Introduction to \LaTeX}, \\
available at \https{www.ctan.org/tex-archive/info/lshort/english/}

The Share\LaTeX{} knowledge base, at \https{www.sharelatex.com/learn}

The \TeX{} Stack Exchange, at \https{tex.stackexchange.com/}
\end{leftfigure}

\section{For typography}

We've spent most of our time here focusing \emph{what} you can do with \LaTeX,
and little on \emph{how} you should use it to create well-designed documents.
Read on:
\begin{leftfigure}
\textit{Practical Typography}, by Matthew Butterick. \\
Available (for free!) at \https{practicaltypography.com}

\textit{Stop Stealing Sheep \& Find Out How Type Works}, by Erik Spiekermann

\textit{Thinking With Type}, by Ellen Lupton

\textit{The Elements of Typographic Style}, by Robert Bringhurst

\textit{Detail in Typography}, by Jost Hochuli
\end{leftfigure}

\backmatter

\setkomafont{chapter}{\Huge\itshape}

% Chicago Manual of Style, Notes & Bib style, ish.
% http://www.chicagomanualofstyle.org/tools_citationguide/citation-guide-1.html
\chapter{Notes}

% Endnotes _mostly_ works... with Koma Script, no less.
% I suppose I can only grumble a little about the hackery below.

% The \llap in \enoteformat throws the endnote number into the margin.
% To compensate, measure about the width of our widest endnote mark
% (see \makeenmark)...
\newlength{\enotewidth}
\settowidth{\enotewidth}{00.\enspace}

% We slapped down our own endnotes heading above with \chapter{Notes},
% so make the package do nothing when it tries.
\renewcommand\enoteheading{}
% Use our normal figures here instead of the superiors we marked endnotes with
% in the body text.
\renewcommand\makeenmark{\theenmark.\enspace}
% TeX hackery lifted from endnotes.sty
% See https://www.tutorialspoint.com/tex_commands/llap.htm
% https://tex.stackexchange.com/q/22852/92465
\renewcommand\enoteformat{\leavevmode\llap{\makeenmark}}
% Indent the whole section by our measured amount much.
\begin{adjustwidth}{\enotewidth}{0pt}
\raggedright
\theendnotes
\end{adjustwidth}

% Redefine cleardoublepage so the Colophon doesn't demand a front page.
% From https://tex.stackexchange.com/a/24068/92465
{\let\cleardoublepage\clearpage \chapter{Colophon}}

This guide was typeset with \LuaLaTeX{} in Garamond Premier by Robert Slimbach.
His revival is based on roman type by
\otford{16}{th} century French
punchcutter Claude Garamond.
Italics are inspired by the work of Garamond's contemporary Robert Granjon.

Monospaced items are set in \texttt{Fira Mono},
originally designed for Mozilla
by Erik Spiekermann and Ralph du~Carrois.
The Fira family draws from Spiekermann's past work on FF~Meta.

Captions use
\href{http://www.fontbureau.com/NHG/}{\textsf{\small Neue Haas Grotesk}},
a Helvetica restoration by Christian Schwartz.
Other digitizations of the ubiquitous Swiss typeface are based on fonts made for
Linotype and phototypesetting machines,
resulting in digital versions with all the compromises and kludges from those
past two generations of printing technology.
Schwartz based his work on Helvetica's original drawings,
producing a design faithful to the original cold metal type.

{\fontspec[Ligatures=TeX, Scale=MatchLowercase]{Futura-Boo}URW Futura}
makes a few guest appearances.
Designed by Paul Renner and first released in 1927,
Futura has found itself almost everywhere,
from advertising and political campaigns to the moon.
Douglas Thomas's recent history of the typeface,
\textit{Never Use Futura}, is a fantastic read.

Various bits of non-Latin text are set in
{\fontspec[Ligatures=TeX,Scale=MatchLowercase]{NotoSerif-Regular}Noto},
a type family by Google that covers \emph{every} language
in the Unicode standard.

Finally,
{\lm Latin Modern}---the OpenType version of Knuth's Computer Modern used throughout
the book---as well
as {\fontspec[Scale=MatchUppercase]{TeX Gyre Termes}\TeX{} Gyre Termes}---the
free alternative to Times Roman seen on page \pageref{typography}---are from
the digital type foundry of Grupa Użytkowników Systemu \TeX{},
the Polish \TeX{} Users' Group.
An overview of their excellent work can be found at the following locations:\\
\http{www.gust.org.pl/projects/e-foundry/latin-modern} \\
\http{www.gust.org.pl/projects/e-foundry/tex-gyre}.
