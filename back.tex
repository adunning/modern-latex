\setlength\parskip{0.8\baselineskip}
\setlength\parindent{0pt}

\chapter{Additional Resources}
\label{resources}

\section{For \texorpdfstring{\LaTeX}{LaTeX}}

As promised at the start, this book is incomplete.
To keep things short,
major \LaTeX{} features---like figures, captions, tables, graphics,
and bibliographies---haven't been discussed.
Use some of these resources to fill in the gaps:

\begin{adjustwidth}{1.5em}{0pt}
The \LaTeX{} Wikibook, at \https{en.wikibooks.org/wiki/LaTeX}

The \TeX{} Stack Exchange, at \https{tex.stackexchange.com/}

\textit{The Not So Short Introduction to \LaTeX}, \\
available at \https{www.ctan.org/tex-archive/info/lshort/english/}

The Share\LaTeX{} knowledge base, at \https{www.sharelatex.com/learn}

\end{adjustwidth}

\section{For typography}

We've spent most of our time here focusing \emph{what} you can do with \LaTeX,
and little on \emph{how} you should use it to create well-designed documents.
Read on:

\begin{adjustwidth}{1.5em}{0pt}
\textit{Practical Typography}, by Matthew Butterick. \\
Available (for free!) at \https{practicaltypography.com}

\textit{Stop Stealing Sheep \& Find Out How Type Works}, by Erik Spiekermann

\textit{Thinking With Type}, by Ellen Lupton

\textit{The Elements of Typographic Style}, by Robert Bringhurst

\textit{Detail in Typography}, by Jost Hochuli
\end{adjustwidth}

\backmatter

\setkomafont{chapter}{\Huge\itshape}

{\raggedright
\renewcommand\makeenmark{\theenmark.\enspace}
% Chicago Manual of Style, Notes & Bib style, ish.
% http://www.chicagomanualofstyle.org/tools_citationguide/citation-guide-1.html
\theendnotes
}

% Redefine cleardoublepage so the Colophon doesn't demand a front page.
% From https://tex.stackexchange.com/a/24068/92465
{\let\cleardoublepage\clearpage \chapter{Colophon}}

This guide was typeset with \LuaLaTeX{} in Garamond Premier by Robert Slimbach.
His revival is based on roman type by
\otford{16}{th} century French
punchcutter Claude Garamond.
Italics are inspired by the work of Garamond's contemporary Robert Granjon.

Monospaced items are set in Matthias Tellen's
\href{https://madmalik.github.io/mononoki/}{\texttt{mononoki}},
a typeface designed to work well on both low-resolution computer monitors
and in high-resolution print.

Captions use
\href{http://www.fontbureau.com/NHG/}{\textsf{\small Neue Haas Grotesk}},
a Helvetica restoration by Christian Schwartz.
Other digitizations of the ubiquitous Swiss typeface are based on fonts made for
Linotype and phototypesetting machines,
resulting in digital versions with all the compromises and kludges from those
past two generations of printing technology.
Schwartz based his work on Helvetica's original drawings,
producing a design faithful to the original cold metal type.

{\fontspec[Ligatures=TeX, Scale=MatchLowercase]{Futura-Boo}URW Futura}
makes a few guest appearances.
Originally released in 1927 by Paul Renner,
Futura has found itself almost everywhere,
from advertising and political campaigns to the moon.
Douglas Thomas's recent history of the typeface,
\textit{Never Use Futura}, is a fantastic read.

Various bits of non-Latin text are set in
{\fontspec[Ligatures=TeX,Scale=MatchLowercase]{NotoSerif-Regular}Noto},
a type family by Google that covers \emph{every} language
in the Unicode standard.

Finally,
{\lm Latin Modern}---the OpenType version of Knuth's Computer Modern used throughout
the book---as well
as {\fontspec[Scale=MatchUppercase]{TeX Gyre Termes}\TeX{} Gyre Termes}---the
free alternative to Times Roman seen on page \pageref{typography}---are from
the digital type foundry of Grupa Użytkowników Systemu \TeX{},
the Polish \TeX{} Users' Group.
An overview of their excellent work can be found at the following locations:\\
\http{www.gust.org.pl/projects/e-foundry/latin-modern} \\
\http{www.gust.org.pl/projects/e-foundry/tex-gyre}.
