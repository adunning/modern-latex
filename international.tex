\chapter{Typographie Internationale}

Surprisingly, many languages besides English exist.
You might want to write with them.

\section{Unicode}

Digitising human language is a complicated topic that has evolved significantly
since \LaTeX's inception, but today,
computers usually represent text using Unicode. Briefly,
\begin{itemize}
\item A Unicode text file is made of a series of \introduce{code points},
    each of which can represent a character to be drawn,
    an accent or other diacritical mark to combine with an adjacent character,
    or some non-printing character,
    such as an instruction to print subsequent text right-to-left.
\item One or more of these code points combines to represent a
    \introduce{grapheme cluster} or \introduce{glyph},
    the shapes within a font that we informally call ``characters''\quotekern.
\begin{centerfigure}
\large%
\fontspec[Ligatures=TeX]{NotoSerif}%
Приве́т
\quad\fontspec[Ligatures=TeX]{NotoSerif-Devanagari}%
नमस्ते
\captionof{figure}{How many characters do you see?
How many code points are they built from?}
\end{centerfigure}
\item Modern font formats contain encoding tables
    which map code points to the glyphs the file contains.
\end{itemize}
\LuaLaTeX{} and \XeLaTeX{} use these tools to render documents
from Unicode input
files.\punckern\footnote{LuaLaTeX accepts \mbox{\acronym{utf}-8} files,
while \XeLaTeX{} is a bit more flexible and also
accepts \mbox{\acronym{utf}-16} and
\mbox{\acronym{utf}-32}.}
Make sure that the fonts you select contain the needed glyphs---many
only provide support for Latin languages.

\section{The polyglossia package}

If your document contains anything besides English,
you should use the \texttt{polyglossia} package.
It will automatically:
\begin{itemize}
\item Load language-specific hyphenation patterns and other typographical
    conventions.
\item Switch between fonts for each language, as specified by the user.
\item Translate document labels,
    like ``chapter''\quotekern, ``section''\quotekern, and so on.
\item Format dates according to language-specific conventions.
\item Format numbers in languages that have their own numbering system
\item Use the \texttt{bidi} package for documents with languages written
    right to left.
\item Set the script and language tags of the current font, if applicable.
\end{itemize}
When using the package, specify the main language of your document,
along with any other languages you use.
Some languages also take regional dialects as an optional argument:
\begin{leftfigure}
\begin{lstlisting}
\usepackage{polyglossia}
\setdefaultlanguage[variant=american]{english}
\setotherlanguage{french}
\end{lstlisting}
\end{leftfigure}
\texttt{polyglossia} will define an environment for each loaded language.
These automatically apply that language's conventions to the text within.
French, for example, places extra space around punctuation, so
\begin{leftfigure}
\begin{lstlisting}
Dexter cried,
\begin{french}
«Omelette du fromage!»
\end{french}
\end{lstlisting}
\end{leftfigure}
gives:\footnote{Yes, it's \emph{omelette au fromage}.
Direct all complaints to Cartoon Network.}
\begin{leftfigure}
Dexter cried,
\begin{french}
«Omelette du fromage!»
\end{french}
\end{leftfigure}

% FFS: https://github.com/reutenauer/polyglossia/issues/68
\directlua{polyglossia.desactivate_frpt()}

\section{What next?}
\begin{itemize}
\item See the \texttt{polyglossia} manual for language-specific commands.
\item Look into the \texttt{babel} package as an alternative to
    \texttt{polyglossia}.\punckern\footnote{\texttt{polyglossia} has better
    support for OpenType font features via \texttt{fontspec}.
    However, \texttt{babel} is a fine substitute if you run into trouble.}
    %as I did in
    %\LuaLaTeX{} (see \url{https://github.com/reutenauer/polyglossia/issues/68}),
\item Try typesetting Japanese or Chinese with the \texttt{xeCJK} or
    \monobox{luatex-ja} packages.
\end{itemize}
