\chapter{A Brief History of \texorpdfstring{\LaTeX}{LaTeX}}
\label{history}

Most serious programmers have heard of Donald Knuth,
the man who coined the term \emph{analysis of algorithms} in 1968
and pioneered many of the computer science fundamentals we use today.
Knuth is perhaps most famous for his ongoing magnum opus,
\textit{The Art of Computer Programming}.

When the first volume of \acronym{taocp} was released that same year,
it was printed the way most books had been since the turn of the century:
with \introduce{hot metal} type.
Each individual letter was cast from molten lead,
then arranged into its line.
These lines were clamped together to form pages of the book,
which were finally inked and pressed against paper.
By March of 1977, Knuth was ready for a second run of \acronym{taocp}, volume~2,
but he was horrified when he received the proofs.
Hot metal typesetting was an expensive, complicated, and time-consuming process,
so publishers had replaced it with phototypesetting,
which works by projecting images of characters onto film.
The new technology, while much cheaper and faster,
didn't provide the same level of quality Knuth had come to
expect.\punckern\endnote{Donald E.~Knuth, \textit{Digital Typography} (Stanford, 1999), 3--5.}

The average author would have resigned themselves to this change and moved on,
but Knuth took great pride in print quality,
especially for the mathematics in his books.
Around this time, he discovered an exciting new technology:
digital typesetting.
Instead of working with metal or film,
letters and shapes were built from tiny dots,
often packed together at over 1,000 per inch.
Inspired by this burgeoning tech and frustrated with the current state of affairs,
Knuth set off on one of the greatest yak shaves\footnote{Programmers
often refer to seemingly unrelated tasks needed to solve their main problem
as ``yak shaving''\quotekern. The phrase is thought to originate from an episode
of \textit{The Ren~\&~Stimpy Show}.\endnote{``yak shaving''\quotekern,
\textit{The Jargon File},
\url{http://www.catb.org/~esr/jargon/html/Y/yak-shaving.html}.}}
of all time.
For years, he paused all work on his books to create his own
digital typesetting system.
When the dust settled in 1978, Knuth had the first version of
\TeX.\punckern\footnote{\TeX\ is pronounced like the first syllable of
``\textbf{tech}nician''\quotekern. It is short for the Greek
{\fontspec[Scale=MatchLowercase]{NotoSerif-Regular}τέχνη},
meaning \introduce{art} or \introduce{craft}.\endnote{Knuth,
\textit{The \TeX book} (Addison-Wesley, 1986), 1.}}

It's hard to appreciate how much of a revolution \TeX\ was,
especially looking back from a time where anybody with a copy
of Word can be their own desktop publisher.
Adobe's \acronym{pdf} wouldn't exist for another decade, so Knuth
invented a device-independent format, \acronym{dvi}.
Scalable fonts were uncommon at the time, so Knuth created a system,
\MF, to rasterize his characters into dots on the
page.\punckern\footnote{In some ways,
\MF\ is more impressive than the PostScript system that
underpins modern TrueType and OpenType font files.
Instead of constructing characters from lines and curves,
\MF\ uses strokes of virtual pens.
By tweaking the parameters of these pens and strokes, one can build entire
font families from the same base designs.}
Perhaps most importantly, Knuth and his graduate students designed algorithms
to automatically hyphenate and justify lines of text into
beautifully-typeset paragraphs.\footnote{These same algorithms went
on to influence the ones Adobe uses in its software today.\punckern\endnote{%
Several sources (\url{http://www.tug.org/whatis.html},
\url{https://tug.org/interviews/thanh.html},
\url{http://www.typophile.com/node/34620})
mention \TeX's influence on the \textit{hz}-program by Peter Karow
and Hermann Zapf, thanks to via Knuth's collaborations with Zapf.
\textit{hz} was later acquired by Adobe and used in part
when creating their paragraph formatting systems for InDesign.}}

\LaTeX{}, short for Lamport~\TeX{}, was developed by Leslie Lamport in the 1980s
as a set of commands for common document layouts.
It was introduced in 1986 with Lamport's guide,
\textit{\LaTeX: A Document Preparation System}.
Other typesetting systems based on \TeX{} also exist,
the other most popular today being Con\TeX{}t.

Development continues today,
both in the form of user-provided packages for \TeX{} and \LaTeX{},
and on improvements to the \TeX{} typesetting program itself.
Currently, there are four versions, or \introduce{engines}:
\begin{description}
\item[\TeX] is the original system by Donald Knuth.
Knuth stopped adding features after version 3.0 in March~1990,
and all subsequent releases have contained only bug fixes.
With each release, the version number asymptotically approaches $\pi$
by adding an additional digit.
The most recent version, 3.14159265, came out in January~2014.

\item[pdf\TeX] is an extension of \TeX{} that provides direct \acronym{pdf}
    output (instead of \TeX's \acronym{dvi}),
    native support for PostScript\footnote{Specifically, PostScript Type 1}
    and TrueType fonts,
    and micro-typographic features discussed in this book's Microtype chapter.
    It was originally developed by
    Hàn Th\rlap{ê}\raisebox{0.2ex}{\,´} Thành
    as part of his PhD thesis
    for Masaryk University in Brno, Czech Republic.\punckern\endnote{%
    Hàn Th\rlap{ê}\raisebox{0.2ex}{\,´} Thành,
    \textit{Micro-typographic extensions to the \TeX{} typesetting system}
    (Masaryk University Brno, October 2000)}

\item[\XeTeX] is a further extension of \TeX{} that adds native support for
    Unicode and modern font formats, such as OpenType.
    It was originally developed by Jonathan Kew in the early 2000s,
    and gained full cross-platform support and inclusion in the \TeX{} Live
    distribution in 2007.\punckern\endnote{Jonathan Kew,
    ``\XeTeX{} Live''\quotekern, \textit{TUGboat} 29, no.~1 (2007)}

\item[\LuaTeX] is similar to \XeTeX{} in its native Unicode and modern font support.
    It also embeds the Lua scripting language into the engine,
    exposing an interface for package and document authors.
    It first appeared in 2007, and is developed by a core team of
    Hans Hagen, Hartmut Henkel, Taco Hoekwater,
    and Luigi Scarso.\punckern\endnote{\url{http://www.luatex.org}}
\end{description}

Building \TeX{} today is an\ldots{} interesting endeavour.
When it was written in the late 1970s,
there were no large, well-documented open-source projects for students to study,
so Knuth set out to make \TeX{} into one.
As part of this effort, \TeX{} was written in a style he calls
\introduce{literate programming}: opposite most programs---where
documentation is interspersed throughout the code---Knuth wrote \TeX{} as a book,
with the code interspersed between paragraphs.
This mix of English and code is called \texttt{WEB}.\punckern\footnote{Knuth
also released a pair of compainion programs named
\texttt{TANGLE} and \texttt{WEAVE}.
The former extracts the book---as \TeX, of course---and the latter
produces the \TeX's Pascal source code.}

Unsurprisingly, most modern systems don't have good tooling for the late 1970s
dialect of Pascal that \TeX{} was written in,
so present-day distributions use another program,
\texttt{web2c}, to convert \TeX{}'s \texttt{WEB} source into C code.
pdf\TeX{} and \XeTeX{} are built by combining the result with other C
and \cpp{} code.
The \LuaTeX{} authors, on the other hand, hand-translated Knuth's Pascal into C,
and have been using the resulting code since 2009.\punckern\endnote{%
Taco Hoekwater, \textit{\LuaTeX says goodbye to Pascal}
(MAPS 39, Euro\TeX{} 2009),
\url{https://www.tug.org/TUGboat/tb30-3/tb96hoekwater-pascal.pdf}}

