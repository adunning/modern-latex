\chapter{Hello, \texorpdfstring{\LaTeX}{LaTeX}!}

Now that you have a \LaTeX{} distribution installed,
let's try it out.
Open up your editor of choice and save the following as \texttt{hello.tex}:
\begin{leftfigure}
\begin{lstlisting}
\documentclass{article}
% Say hello
\begin{document}
Hello, World!
\end{document}
\end{lstlisting}
\end{leftfigure}
Next, we'll run this file through \LaTeX{} (the program)\footnote{Not to be
confused with \LaTeX{} the lunchbox, \LaTeX{} the breakfast cereal,
or \LaTeX{} the flamethrower. The kids love this stuff!}
to get our document.
The installation placed several different versions---or
\introduce{engines}---on your machine,
but for this entire book, we'll always use \LuaLaTeX{} or \XeLaTeX.
These are the newest engines available---see Appendix~\ref{history} for an
explanation of how they differ from the others.

If you are using a \LaTeX{}-specific editor, it probably contains a drop-down
menu or some other means to select the engine you'd like to use,
as well as a button to generate your document.
Otherwise, from a terminal,\punckern\footnote{How to work a terminal emulator,
making sure the newly-installed \LaTeX{} programs are in your \texttt{PATH},
and so on are all outside the scope of this book.
As is tradition, the leading dollar sign in examples just denotes a console
prompt, and shouldn't actually be typed.}
run the following:
\begin{leftfigure}
\begin{lstlisting}
$ xelatex hello.tex
\end{lstlisting}
\end{leftfigure}
Feel free to use \texttt{lualatex} instead---there are a few differences
between the two that we'll discuss later, but either is fine for now.
With luck, you should see some output that ends in a message like:
\begin{leftfigure}
\begin{lstlisting}
Output written on hello.pdf (1 page).
Transcript written on hello.log.
\end{lstlisting}
\end{leftfigure}
And in your current directory, you should find a newly minted \texttt{hello.pdf}.
Open it up and you should see a page with this at the top:
\begin{leftfigure}
\lm Hello, World!
\end{leftfigure}
Congrats,
you just created your first document!
Let's unpack what we did.

All \LaTeX{} documents begin with a \verb|\documentclass| declaration,
which picks a base ``style'' to use.
Many classes are available---and you can even create your own---but common
ones include \texttt{article}, \texttt{report}, \texttt{book},
and \texttt{beamer}.\punckern\footnote{This last one is for slideshows.
The name is the German term for a projector.}
For your average document, \texttt{article} is probably a good choice.
The next line, \verb|% Say hello|,
is a \introduce{comment}---anything placed after a percent symbol on a line
is ignored by the engine,
so we can use \texttt{\%} to leave notes for anybody reading
the document's source.\punckern\footnote{Including, perhaps most importantly,
a confused version of your future self!}
Finally, we use \verb|\begin{document}| to tell \LaTeX{} that what follows
is our actual contents,
and we use \verb|\end{document}| to state that we are finished.

Let's cover some more basics.

\section{Spacing}

\LaTeX{} generally handles inter-word spacing for you, regardless of how many
times you hit the space bar.\punckern\footnote{Or the tab key}
For example,
\begin{leftfigure}
\begin{lstlisting}
The number  of   spaces    between words doesn't   matter. The same
is true for sentences.

An empty line ends the previous paragraph and starts a new one.
\end{lstlisting}
\end{leftfigure}
yields
\begin{leftfigure}
\lm The number  of   spaces    between words doesn't   matter. The same
is true for sentences.

An empty line ends the previous paragraph and starts a new one.
\end{leftfigure}
Notice that \LaTeX{} automatically follows typographic
conventions, such as indenting new paragraphs and leaving more space after a
period than it leaves between words.
One quirk to be aware of is that comments ``eat'' all of the leading
space on the subsequent line, such that
\begin{leftfigure}
\begin{lstlisting}
This% weird, right?
  is strange
\end{lstlisting}
\end{leftfigure}
gives
\begin{leftfigure}
This% weird, right?
  is strange
\end{leftfigure}

\section{Commands}

\LaTeX{} provides various commands to issue layout instructions to
the engine, and you can define your own as well.
Their names always begin with a backslash (\,\texttt{\textbackslash}\,),
contain only letters, and are case-sensitive.\punckern\footnote{%
% \verb doesn't play nicely in footnotes, so...
\texttt{\textbackslash foo}
is different from \texttt{\textbackslash Foo}, for example.}
Some commands require parameters, or \introduce{arguments}---\verb|\documentclass|,
for example, needs to know which class we want.
Arguments are enclosed in subsequent pairs of braces,
so if some command needed two arguments, we would type:
\begin{leftfigure}
\begin{lstlisting}
\somecommand{argument1}{argument2}
\end{lstlisting}
\end{leftfigure}
Many commands also take optional arguments, which precede the mandatory ones,
are enclosed in square brackets, and are separated by commas.
Say you want to inform \LaTeX{} that your document will printed as double-sided
pages\footnote{\texttt{twoside} introduces commands
that only make sense in the context of double-sided printing,
such as ones that skip to the start of the next odd page.
It also allows you to have different margins for even and odd pages,
which is useful for texts like this book.}
in 11~point
type.\punckern\footnote{We'll discuss font sizes and much more in
\chapref{formatting}.}
This is done with optional arguments to \verb|\documentclass|:
\begin{leftfigure}
\begin{lstlisting}
\documentclass[11pt, twoside]{article}
\end{lstlisting}
\end{leftfigure}

Other commands take no arguments at all, such as \verb|\LaTeX|,
which prints the \LaTeX{} logo.
These commands consume any space that follows them.
For example,
\begin{leftfigure}
\begin{lstlisting}
\LaTeX is great, but it can be a bit odd sometimes.
\end{lstlisting}
\end{leftfigure}
will give you
\begin{leftfigure}
\lm \LaTeX is great, but it can be a bit odd sometimes.
\end{leftfigure}
You can fix this by adding an empty pair of braces to the command.
Of course, the braces aren't needed if there is no space to preserve:
\begin{leftfigure}
\begin{lstlisting}
Let's learn \LaTeX! \LaTeX{} is a powerful tool,
but a few of its rules are a little weird.
\end{lstlisting}
\end{leftfigure}
gets us
\begin{leftfigure}
\lm Let's learn \LaTeX! \LaTeX{} is a powerful tool,
but a few of its rules are a little weird.
\end{leftfigure}

\section{Special characters and line breaks}

Some characters have special meanings in \LaTeX.
We saw above, for example, that \verb|%| starts a comment
and \verb|\| starts a command.
The full list of special characters is:
\begin{leftfigure}
\begin{lstlisting}
# $ % ^ & _ { } ~ \
\end{lstlisting}
\end{leftfigure}
Each has a corresponding command
to print it in your document:
\begin{leftfigure}
\begin{lstlisting}
\# \$ \% \^{} \& \_ \{ \} \~{} \textbackslash
\end{lstlisting}
\end{leftfigure}
will produce
\begin{leftfigure}
\lm \# \$ \% \^{} \& \_ \{ \} \~{} \textbackslash
\end{leftfigure}
Regardless of what comes after them, you must always add braces to
the caret (\,\texttt{\^{}}\,) and tilde (\,\~{}\,).
This is a relic from the days when these commands were used to produce
\introduce{diacritical marks}:
once upon a time, users had to typeset ``jalapeño'' as
\verb|jalape\~no|.
Today, we just type \texttt{ñ} into our source
file.\punckern\footnote{The ease with which you can do this depends
on the keyboard you're using, the language settings in your \acronym{os},
and your editor.
Later on, we'll talk much more about non-English languages and Unicode fun.}

If you're wondering why we print \texttt{\textbackslash} with
\verb|\textbackslash| instead of \verb|\\|,
it's because the latter is the command to force a line break.
\begin{leftfigure}
\begin{lstlisting}
Give me \\
a brand new line!
\end{lstlisting}
\end{leftfigure}
obeys:
\begin{leftfigure}
\lm Give me \\
a brand new line!
\end{leftfigure}
Use this power judiciously---deciding how to best break paragraphs into lines
is one of \LaTeX{}'s greatest skills.

\section{Environments}

We often format text in \LaTeX{} by placing it into \introduce{environments}.
These always start with \verb|\begin{name}| and conclude with \verb|\end{name}|,
where \texttt{name} is that of the desired environment.
Take the \texttt{quote} environment,
which adds additional space on both sides of a block quotation:
\begin{leftfigure}
\begin{lstlisting}
Donald Knuth once wrote,
\begin{quote}
We should forget about small efficiencies,
say about 97\% of the time:
premature optimization is the root of all evil.
Yet we should not pass up our opportunities in that critical 3\%.
\end{quote}
\end{lstlisting}
\end{leftfigure}
quotes
\begin{leftfigure}
\lm
Donald Knuth once wrote,
\begin{quote}
We should forget about small efficiencies,
say about 97\% of the time:
premature optimization is the root of all evil.
Yet we should not pass up our opportunities in that critical 3\%.
\end{quote}
\end{leftfigure}

\section{Groups and command scope}
Some commands change how \LaTeX{} typesets the following text.
\verb|\itshape|, for example, \textit{italicizes} everything that comes after it.
To limit a command's influence to a certain region, we surround it with braces.
\begin{leftfigure}
\begin{lstlisting}
{\itshape Sometimes we want italics}, but only sometimes.
\end{lstlisting}
\end{leftfigure}
is set as
\begin{leftfigure}
\lm {\itshape Sometimes we want italics}, but only sometimes.
\end{leftfigure}
The text within the braces forms a \introduce{group},
and commands issued inside the group only take effect until it ends.
Environments also form implicit groups:
\begin{leftfigure}
\begin{lstlisting}
\begin{quote}
\itshape If I italicize a quote, the following text will
use upright type again.
\end{quote}
See? Back to normal.
\end{lstlisting}
\end{leftfigure}
produces
\begin{leftfigure}
\lm
\begin{quote}
\itshape If I italicize a quote, the following text will
use upright type again.
\end{quote}
See? Back to normal.
\end{leftfigure}
Another use of groups is to address the spacing oddities of zero-argument
commands: some prefer \verb|{\LaTeX}| over \verb|\LaTeX{}|.
