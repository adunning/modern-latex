\chapter{Layout}

\section{Justification and alignment}

\LaTeX{} justifies text outstandingly well.
Instead of considering each line individually---as most word processors and
web browsers do---it examines all possible line breaks in a given paragraph,
then chooses those that give the best overall
spacing.\punckern\endnote{Donald E.~Knuth and Michael F.~Plass,
\textit{Breaking Paragraphs Into Lines} (Stanford, 1981)}
Combined with its ability to automatically hyphenate words,
which permits line breaks in many more places,\punckern\endnote{%
Franklin Mark Liang,
\textit{Word Hy-phen-a-tion by Com-put-er} (Stanford, 1983),
\http{www.tug.org/docs/liang/}}
this produces better paragraph layouts than almost any other software.

But sometimes we don't want our text justified.
If you would like it to be flush left,
place it in a \texttt{flushleft} environment
or add \verb|\raggedright| to the current group.
To center it, place it in a
\texttt{center} environment or add \verb|\centering| to the current group.
And to flush it against the right margin,
use a \texttt{flushright} environment or \verb|\raggedleft|.

\begin{leftfigure}
\begin{lstlisting}
\begin{flushleft}
This text is flush left with a ragged right edge.
Some prefer this layout since inter-word spacing
is more consistent than it is in justified text.
\end{flushleft}

\begin{center}
This text is centered.
\end{center}

\begin{flushright}
And this text is flush right.
\end{flushright}
\end{lstlisting}
\end{leftfigure}
sets
\begin{leftfigure}
\fbox{%
\begin{minipage}{0.7\textwidth}
\lm%
\begin{flushleft}
This text is flush left with a ragged right edge.
Some prefer this layout since inter-word spacing
is more consistent than it is in justified text.
\end{flushleft}

\begin{center}
This text is centered.
\end{center}

\begin{flushright}
And this text is flush right.
\end{flushright}
\end{minipage}
}
\end{leftfigure}

\section{Lists}

\LaTeX{} provides three environments for lists:
\texttt{itemize}, \texttt{enumerate}, and \texttt{description}.
In all three, items starts with \verb|\item| commands.
The \texttt{itemize} environment uses bullets.
With:
\begin{leftfigure}
\begin{lstlisting}
\begin{itemize}
\item 5.56 millimeter
\item 9 millimeter
\item 7.62 millimeter
\end{itemize}
\end{lstlisting}
\end{leftfigure}
you get
\begin{leftfigure}
\lm%
\begin{itemize}[leftmargin=*]
\item 5.56 millimeter
\item 9 millimeter
\item 7.62 millimeter
\end{itemize}
\end{leftfigure}

\bigskip
\noindent The \texttt{enumerate} environment numbers your list:
\begin{leftfigure}
\begin{lstlisting}
\begin{enumerate}
\item Collect underpants
\item ?
\item Profit
\end{enumerate}
\end{lstlisting}
\end{leftfigure}
produces
\begin{leftfigure}
\lm%
\begin{enumerate}[leftmargin=*]
\item Collect underpants
\item ?
\item Profit
\end{enumerate}
\end{leftfigure}

\bigskip
\noindent The \texttt{description} environment starts each item with some
emphasized \introduce{label},
then indents lines that follow:
\begin{leftfigure}
\begin{lstlisting}
\begin{description}
\item[Alan Turing] was a British mathematician who laid much
    of the groundwork for the field of computer science.
    He is perhaps most remembered for his model of computation,
    the Turing machine.
\item[Edsger Dijkstra] was a Dutch computer scientist.
    His contributions in many subdomains---such as concurrency
    and graph theory---are still in wide use today.
\item[Leslie Lamport] is an American computer scientist.
    He defined the concept of sequential consistency,
    which is used to safely communicate between tasks
    running in parallel on multiple processors.
\end{description}
\end{lstlisting}
\end{leftfigure}
gives us
\begin{leftfigure}
\lm%
\begin{minipage}{0.9\textwidth}
\begin{description}[leftmargin=*]
\item[\lm Alan Turing] was a British mathematician who laid much
    of the groundwork for the field of computer science.
    He is perhaps most remembered for his model of computation,
    the Turing machine.
\item[\lm Edsger Dijkstra] was a Dutch computer scientist.
    His contributions in many subdomains---such as concurrency
    and graph theory---are still in wide use today.
\item[\lm Leslie Lamport] is an American computer scientist.
    He defined the concept of sequential consistency,
    which is used to safely communicate between tasks
    running in parallel on multiple processors.
\end{description}
\end{minipage}
\end{leftfigure}

\section{Columns}

We often split layouts into several columns, especially when printing on
\textsc{a4} or \acronym{us}~\textsc{l}etter paper,
since it allows more comfortable line widths at standard
8--12\,pt text sizes.\punckern\footnote{You'll see different advice depending
on where you look, but a rule of thumb is to design layouts to have
fewer than 90 characters (including spaces) per line.
Shorter lines keep readers from needing to scan very far to find the start of
the next line.
This produces a more comfortable experience.}
You can either add the \texttt{twocolumn} option to your document class,
which splits everything in two, or you can use the \texttt{multicols}
environment from the \texttt{multicol} package:
\begin{leftfigure}
\begin{lstlisting}
One nice feature of \texttt{multicol} is that you can
combine arbitrary layouts.
\begin{multicols}{2}
In this example we start with one column,
then create a short section with two.
The package ensures that the text inside is split
so that each column is about the same height.
\end{multicols}
\end{lstlisting}
\end{leftfigure}
is split into
\begin{leftfigure}
\lm%
One nice feature of \texttt{multicol} is that you can
combine arbitrary layouts.
\begin{multicols}{2}
In this example we start with one column,
then create a short section with two.
The package ensures that the text inside is split
so that each column is about the same height.
\end{multicols}
\end{leftfigure}

\section{Page breaks}

Some commands, like a book's \verb|\chapter|, insert page breaks.
You can add your own with \verb|\clearpage|.
If you are using the \texttt{twoside} document class option for double-sided
printing, you can break to the front of the next page with
\verb|\cleardoublepage|.

\section{Footnotes}

Footnotes can be used to insert remarks which readers might
find useful, but aren't crucial to the main text.
The \verb|\footnote| command places a marker at its location in the
body text, then sets its argument as a footnote at the bottom of the current
page:
\begin{leftfigure}
\begin{lstlisting}
I love footnotes!\footnote{Perhaps a bit too much\dots}
\end{lstlisting}
\end{leftfigure}
proclaims
\begin{leftfigure}
\lm%
I love footnotes!\footnote{Perhaps a bit too much\dots}
\end{leftfigure}

\section{What next?}
\begin{itemize}
\item Control paragraph spacing, either using the relevant
KOMA~Script options, or with the \texttt{parskip} package.
\item Set the page size and margins with the \texttt{geometry} package.
\item Customize list formatting with the \texttt{enumitem} package.
\item Create tables with the \texttt{tabular} environment.
\item Align text with tab stops using the \texttt{tabbing} environment.
\item Choose footnote symbols and layout with KOMA~Script and the
    \texttt{footmisc} package.
\item Insert horizontal and vertical space with commands like
    \verb|\vspace|, \verb|\hspace|, \verb|\vfill|, and \verb|\hfill|.
\item Learn what units \LaTeX{} provides for specifying spacing. \\
    (We've already mentioned a few here, such as
    \texttt{pt}, \texttt{bp}, and \texttt{in}.)
\end{itemize}
