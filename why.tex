\chapter{\texorpdfstring{\LaTeX}{LaTeX}?}

\LaTeX{} is a program\footnote{And a markup language,
and (sort of) a programming language, and a bit of a cult.
But we'll get to all of that.}
for creating written documents, such as papers, presentations,
or even the book you're reading right now.
In some sense, it's an alternative to the likes of Microsoft Word,
Apple Pages, Google Docs, or LibreOffice Writer.

All of these other desktop publishing systems are
\introduce{What You See Is What You Get} \acronym{(wysiwyg)} systems---whatever
you see on screen is what the final document will look like.
This is often incredibly convenient, since many changes to your document
can be made with a few clicks of the mouse,
but it isn't without its downsides.
In a \acronym{wysiwyg} system,
you must constantly concern yourself with both
the \emph{content} of your text and its \emph{layout}.
Keeping the look of a large document consistent is hard---if you increased
the size of one caption, did you make sure to find all the other
captions and do the same?\footnote{This isn't to say that it's impossible
to creat nice-looking documents with these tools.
Many of them have gotten much better at automating these sorts of
tasks in recent years.}
If the editor formats something in a way you didn't intend,
how difficult is it to change?%\footnote{I spent far too much of my childhood
%fighting with Word about how it wrapped text around images.}

\LaTeX\footnote{Pronounced ``lay-tech'' or ``lah-tex''} is different.
Here, you write your document as a ``plain'' text file,
using \introduce{markup} to specify how things should look.
When you are ready,
\LaTeX{} generates your document from this file
based on the rules you've given it.
If you've ever done any web development, this is a familiar process.
Just as \acronym{html} and \acronym{css} describe the web page you want
browsers to render, \LaTeX{} describes your document.

\begin{figure}[h]
\centering
\begin{lstlisting}
\LaTeX\footnote{Pronounced ``lay-tech'' or ``lah-tex''} is different.
Here, you write your document as a ``plain'' text file,
using \introduce{markup} to specify how things should look.
When you are ready,
\LaTeX{} generates your document from this file
based on the rules you've given it.
If you've ever done any web development, this is a familiar process.
Just as \acronym{html} and \acronym{css} describe the web page you want
browsers to render, \LaTeX{} describes your document.
\end{lstlisting}
\caption{The \LaTeX{} markup that built the above paragraph}
\end{figure}

This might seem alien to you if you've never worked with a markup system before.
However, it comes with a few advantages:
\begin{enumerate}
\item You can think separately about your document's contents and
    its presentation, how it looks.
    \LaTeX{} automatically ensures that fonts, sizes, line heights,
    and other layout considerations are consistent according to the rules you set.
\item You can define your own rules, and tweak them to immediately change
    every place they're used in your document.
    For example, the \verb|\introduce| and \verb|\acronym| commands you saw above
    are my own creations. The former currently \introduce{italicizes} text, and
    the latter sets things in \acronym{small caps} with a bit of extra
    \,\textsc{\addfontfeature{LetterSpace=15}letterspacing}\, so that things don't
    look \textsc{too crowded}.
    If I wake up tomorrow and decide that new terms should look
    \textbf{\itshape like this} and acronyms
    {\small\addfontfeature{LetterSpace=8} LIKE THIS},
    I just change the two lines that define those commands,
    and every single place in the book that uses them takes on those new styles.
\item In \LaTeX, the document's structure is immediately visible,
    and can be easily copied.
    (Compare this to \acronym{wysiwyg} systems, where it is often not obvious
    how certain formatting was produced,
    or how to emulate it.)
\item Since the document source is plain text,
    \begin{itemize}
    \item Document sources can be read and understood with any text editor.
    \item Content can be easily generated programmatically.
    \item Changes can be easily tracked with version control software.
    \end{itemize}
\end{enumerate}

This is all well and good,
but misses the main selling point of \LaTeX.
You should try it because you care about\ldots

\chapter{Typography}

Modern life is a constant battle for your attention---every day,
dozens of ads, apps, emails, sites, and texts fight
for a few short minutes of your time.
To put ideas into the world,
nothing is more important than catching and holding
the attention of your audience.
Typography is a tool to do so---a good design doesn't ``look nice''
only for the sake of art---it draws readers in.\punckern\endnote{Matthew Butterick,
\textit{Typography for Docs}, Write The Docs Conference, April 8, 2013}
%Matthew Butterick's \textit{Practical Typography} calls it
%``the visual component of the written word.\quotekern''
It's the \emph{how} of text.

\begin{adjustwidth}{2em}{0pt}
\fontspec{TeX Gyre Termes}\fontsize{12pt}{24pt}\selectfont\raggedright
\noindent Typography is why all the terrible essays you wrote in school
looked like this.
Would you prefer to read an entire book this way?
\end{adjustwidth}
\bigskip

\noindent It is why

\bigskip
\begin{adjustwidth}{2em}{0pt}
\noindent\fontspec{FuturaCon-ExtBol}\Large DO IT LATER
\end{adjustwidth}
\bigskip

\noindent might remind you of a certain shoe company's advertising,
or how

\begin{center}
\fontspec[Ligatures=TeX, Scale=MatchLowercase]{Futura-Med}
\noindent MEN FROM THE PLANET EARTH \\
FIRST SET FOOT UPON THE MOON \\
JULY 1969, A.~D.
\end{center}

\noindent Typography tells people
what you have to say before they read a single word.

This is where \LaTeX{} shines: unless you want to give Adobe large sums
of money for InDesign or InCopy,
you won't find any better typesetting software.
By carefully handling subtle details---like how paragraphs are broken into lines,
or how words are spaced and hyphenated---\LaTeX{} provides high-quality layout
with almost no effort from you, the author.
Modern versions can also take advantage of new\footnote{By new,
I mean ``from the mid-1990s''\quotekern, but other desktop publishing software and web
browsers are only just starting to catch up.} technologies in computer typography,
such as OpenType fonts, giving you the same tools available to professional
designers and publishers.

\chapter{Another Guide?}
