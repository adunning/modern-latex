\chapter{Typography and You}
\label{typography}

Modern life is a constant battle for your time---every day,
dozens of ads, apps, clips, emails, sites, shows, and texts fight
for a few short minutes of it.
To put ideas into the world,
nothing is more important than catching and holding
the attention of your audience.
When we write, we don't just reach readers with the words we use.
We reach them with \introduce{typography}: how those words appear.
Good design doesn't ``look nice''
only for the sake of art---it draws readers
in.\punckern\endnote{Matthew Butterick, ``Typography for Docs''
(presented at the Write The Docs Conference, April 8, 2013),
\url{https://www.youtube.com/watch?v=8J6HuvosP0s}}
%Matthew Butterick's \textit{Practical Typography} calls it
%``the visual component of the written word.\quotekern''
It sets their expectations and establishes a subliminal brand for your
work.\punckern\endnote{Erik Spiekermann, ``Type is Visible Language''
(presented at Beyond Tellerrand, Düsseldorf, Germany, May 19--21, 2014),
\url{https://www.youtube.com/watch?v=ggQpDu63kk0}}
\begin{leftfigure}
\fontspec{TeX Gyre Termes}\fontsize{12pt}{24pt}\selectfont\raggedright
Typography is why this reminds you of terrible essays
you wrote in school.
Do you see many books that look like this?
Why do you think that might be?
\end{leftfigure}
\medskip

\noindent It is why
\begin{leftfigure}
\fontspec{FuturaCon-ExtBol}\Large DO IT LATER
\end{leftfigure}
seems oddly reminiscent of a certain shoe company's advertising,
or how
\begin{center}
\fontspec[Ligatures=TeX, Scale=MatchLowercase]{Futura-Med}
MEN FROM THE PLANET EARTH \\
FIRST SET FOOT UPON THE MOON \\
JULY 1969, A.~D.
\end{center}

If you care about more than \emph{what} you tell your readers, but \emph{how}
you tell it, you should try \LaTeX.\punckern\footnote{Pronounced ``lay-tech''
or ``lah-tech''}
It's a program\footnote{And a markup language,
and (sort of) a programming language, and a bit of a cult.
But that's all for later.}
for crafting written documents, like papers, presentations,
or even the book you're reading right now.
And unless you want to give Adobe large sums
of money for InDesign or InCopy,
you won't find any better typesetting software.
By carefully handling subtle details,
\LaTeX{} provides high-quality layout
with relatively little effort from you, the author.
Modern versions can also take advantage of new\footnote{By new,
I mean ``from the mid-1990s''\quotekern, but web browsers and desktop publishing
software are only just starting to catch up.} advances in computer typography,
giving you the same tools used by professional designers and publishers.

\section{\texorpdfstring{\LaTeX}{LaTeX}?}
\LaTeX{} is an alternative to ``word processors'' like
Microsoft Word, Apple Pages, Google Docs, and LibreOffice Writer.
Unlike these other applications, which work on the principle of
\introduce{What You See Is What You Get} \acronym{(wysiwyg)},
a \LaTeX{} document is written as a ``plain'' text file,
using \introduce{markup} to specify how things should look.
\LaTeX{} then takes this source file and builds your document
based on the formatting rules you give it.
If you've done any web development, this is a similar process---just
as \acronym{html} and \acronym{css} describe the page you want
browsers to draw, your markup describes the appearance of your
document to \LaTeX.

\begin{leftfigure}
\begin{lstlisting}
\LaTeX{} is an alternative to ``word processors'' like
Microsoft Word, Apple Pages, Google Docs, and LibreOffice Writer.
Unlike these other applications, which work on the principle of
\introduce{What You See Is What You Get} \acronym{(wysiwyg)},
a \LaTeX{} document is written as a ``plain'' text file,
using \introduce{markup} to specify how things should look.
\LaTeX{} then takes this source file and builds your document
based on the formatting rules you give it.
If you've done any web development, this is a similar process---just
as \acronym{html} and \acronym{css} describe the page you want
browsers to draw, your markup describes the appearance of your
document to \LaTeX.
\end{lstlisting}
\captionof{figure}{The \LaTeX{} markup for the above paragraph}
\end{leftfigure}
\vfill

This might seem alien to you if you've never worked with a markup system before.
However, it comes with a few advantages:
\begin{enumerate}
\item You can handle a document's contents and its presentation separately.
    At the beginning of each document,
    you describe the design you want.
    \LaTeX{} takes it from there,
    keeping fonts, sizes, layout,
    and other formatting considerations consistent across your
    whole text.
    Compare this to a \acronym{wysiwyg} system,
    where you must constantly concern yourself with your work's appearance
    as you write.
    If you changed the look of a caption,
    did you make sure to find all the other captions and do the
    same?
    If the program formats something in a way you don't like,
    how hard is it to fix?%\footnote{I spent far too much of my childhood
    %fighting with Word about how it wrapped text around images.}

\item You can define your own rules, then tweak them to immediately change
    everywhere they're used in your document.
    For example, the \verb|\introduce| and \verb|\acronym| commands you saw above
    are my own creations. The former \introduce{italicizes} text, and
    the latter sets words in \acronym{small caps} with a bit of extra
    \,\textsc{\addfontfeature{LetterSpace=15}letterspacing}\, so the characters
    don't look \textsc{too crowded}.
    If I wake up tomorrow and decide to introduce new terms
    \textbf{\itshape with this look} and set acronyms
    {\small\addfontfeature{LetterSpace=8} LIKE THIS},
    I just change the two lines that define those commands
    to update every place in the book that uses them.

\clearpage
\item Since the document source is plain text,
    \begin{itemize}
    \item It can be read and understood with any text editor.
    \item A document's structure is immediately visible
        and easily replicated.\punckern\footnote{Compare this to
        \acronym{wysiwyg} systems, where it is often unobvious
        how certain formatting was produced or how to replicate it.}
    \item Content can be generated programmatically.
    \item You can track changes with standard version control software.
    \end{itemize}
\end{enumerate}

\section{Another guide?}

One might wonder why the world needs another \LaTeX{} guide.
After all, it's been out for more than 30 years.
A quick Amazon search finds nearly a dozen books on the topic.
There are plenty of great resources online.

Unfortunately, most guides have two fatal flaws: they are long,
and they are old.
The first is anathema to newcomers---if somebody asks about \LaTeX{},
throwing a 200+ page book at them isn't an encouraging start.
The second matters because of how much the world of computer typesetting has
changed since 1986.

When \LaTeX{} was first released that year, none of the publishing technologies
we use today existed.
Adobe wouldn't introduce their Portable Document Format for another seven years.
Digital typography was a new field, and desktop publishing was \emph{just}
getting off the ground.\punckern\endnote{\textit{Graphic Means:
A History of Graphic Design Production}, directed by Briar Levit (2017)}
This shows---badly---in most \LaTeX{} guides.
If you look for instructions to change your document's font,
you'll likely get swamped with bespoke nonsense.\punckern\footnote{%
Take all criticisms of \LaTeX's past here with a grain of
salt. After all, the fact that all of the technology around it became
obsolete---multiple times---is a testament to its staying power.}

The good news is that  \LaTeX{} has improved by leaps and bounds in recent years.
It's time for a guide that doesn't weigh you down with decades of legacy
or try to be a comprehensive reference.
After all, you're a smart, resourceful individual who knows how to use a search
engine.
This book will:
\begin{enumerate}
\item Teach you the very basics of using \LaTeX.
\item Point you to places where you can learn more.
\item Show you how to use modern technologies like OpenType and microtypography
    to create professional-quality documents.
\item End promptly thereafter.
\end{enumerate}
Let's begin.
