\chapter{Typography and You}
\label{typography}

Modern life is a constant battle for your time,
fought by endless hordes of ads, apps, clips, emails, sites, shows, and texts.
To put ideas into the world amidst this din,
nothing matters more than seizing
your audience's attention.
And when we write, we don't just reach readers with our words.
We grab them with \introduce{typography}: how those words appear.
Good design isn't mere art---it's function.
It draws readers in,
informs their expectations, and builds a subliminal brand for your
work.\punckern\endnote{Erik Spiekermann, ``Type is Visible Language''
(presented at Beyond Tellerrand, Düsseldorf, Germany, May 19--21, 2014),
\https{www.youtube.com/watch?v=ggQpDu63kk0}}
% Smoke test: a line should be 2-3 alphabets wide
%\\ abcdefghijklmnopqrstuvwxyzabcdefghijklmnopqrstuvwxyzabcdefghijklmnopqrstuvwxyz
\begin{leftfigure}
\fontspec{TeX Gyre Termes}\fontsize{12bp}{24bp}\selectfont\raggedright
Typography is why this sentence evokes memories of awful essays
you wrote in school.
Do many books look this way? Why not?
\end{leftfigure}
\medskip

\noindent It is why
\begin{leftfigure}
\fontspec{FuturaCon-ExtBol}\Large DO IT LATER
\end{leftfigure}
seems oddly reminiscent of a shoe company named after a Greek goddess.
With its help,\punckern\endnote{Douglas Thomas,
``Over the Moon for Futura''\quotekern,
\textit{Never Use Futura}
(New York, 2017)}
two men landed on a dusty alien plain, carrying a plaque that read:
\begin{center}
\fontspec[Ligatures=TeX, Scale=MatchLowercase]{Futura-Med}
HERE MEN FROM THE PLANET EARTH \\
FIRST SET FOOT UPON THE MOON \\
JULY 1969, A.~D. \\
WE CAME IN PEACE FOR ALL MANKIND
\end{center}

If you care not only about \emph{what} you tell your readers, but \emph{how}
you tell them, you should try \LaTeX.\punckern\footnote{Pronounced ``lay-tech''
or ``lah-tech''}
It's a program\footnote{And a markup language,
and (sort of) a programming language.
More on that later.}
for crafting written documents, like essays, papers, presentations,
or even the book you're reading right now.
And unless you give Adobe large sums
of money for InDesign or InCopy,
you won't find any equals.
By carefully handling subtle details,
\LaTeX{} can produce high-quality typesetting
with little effort from you, the author.
Modern versions can also leverage recent\footnote{By recent,
I mean ``from the mid-1990s''\quotekern, but web browsers and desktop publishing
software are only just starting to catch up.} advances in computer typography,
offering you the same tools used by professional designers and publishers.

\section{\texorpdfstring{\LaTeX}{LaTeX}?}

\LaTeX{} is an alternative to ``word processors'' like
Microsoft Word, Apple Pages, Google Docs,
and LibreOffice Writer.
These other applications operate on the principle of
\introduce{What You See Is What You Get}
\acronym{(wysiwyg)}, where what's on screen is the same
as what comes out of your printer.
\LaTeX{} is different. Here, documents are written as
``plain'' text files, using \introduce{markup} to specify
how they should look.
If you've done any web development, this is a similar
process---just as \acronym{html} and \acronym{css} describe
the page you want browsers to draw, your markup describes
the appearance of your document to \LaTeX.

\begin{samepage}
\begin{leftfigure}
\begin{lstlisting}
\LaTeX{} is an alternative to ``word processors'' like
Microsoft Word, Apple Pages, Google Docs,
and LibreOffice Writer.
These other applications operate on the principle of
\introduce{What You See Is What You Get}
\acronym{(wysiwyg)}, where what's on screen is the same
as what comes out of your printer.
\LaTeX{} is different. Here, documents are written as
``plain'' text files, using \introduce{markup} to specify
how they should look.
If you've done any web development, this is a similar
process---just as \acronym{html} and \acronym{css} describe
the page you want browsers to draw, your markup describes
the appearance of your document to \LaTeX.
\end{lstlisting}
\captionof{figure}{The \LaTeX{} markup for the above paragraph}
\end{leftfigure}
\end{samepage}

This might seem alien to you if you've never worked with a markup system before.
However, it comes with a few advantages:
\begin{enumerate}
\item You can handle your work's content and its presentation separately.
    At the start of each document,
    you describe the design you want.
    \LaTeX{} takes it from there, maintaining consistent
    fonts, sizes, page layouts, and other formatting concerns across your
    whole text.
    Compare this to a \acronym{wysiwyg} system,
    where you constantly deal with appearances
    as you write.
    If you changed the look of a caption,
    were you sure to find all the other captions and do the
    same?
    If the program formats something in a way you don't like,
    is it hard to fix?%\footnote{I spent far too much of my childhood
    %fighting with Word about how it wrapped text around images.}

\item You can define your own rules, then tweak them to instantly adjust
    every place they're used.
    For instance, the \verb|\introduce| and \verb|\acronym| commands seen above
    are my own creations. The former \introduce{italicizes} text, and
    the latter sets words in \acronym{small caps} with a bit of extra
    \,\textsc{\addfontfeature{LetterSpace=15}letterspacing}\, so the characters
    don't look \textsc{\addfontfeature{LetterSpace=-2}too crowded}.
    If tomorrow I decide to introduce new terms
    \textbf{\itshape with this look} and set acronyms
    {\small\addfontfeature{LetterSpace=6} LIKE THIS},
    I just change the two lines that define those commands,
    and every spot in this book that uses them is immediately updated.

\item Since the document source is plain text,
    \begin{itemize}
    \item It can be read and understood with any text editor.
    \item Structure is immediately visible
        and easily replicated.\punckern\footnote{Compare this to
        \acronym{wysiwyg} systems, where it is often unobvious
        how certain formatting was produced or how to replicate it.}
    \item Content can be generated by scripts and programs.
    \item Changes can be tracked with standard version control software.
    \end{itemize}
\end{enumerate}

\section{Another guide?}

You might wonder why the world needs another \LaTeX{} guide.
After all, it's been out for decades.
A quick Amazon search finds nearly a dozen books on the topic.
There are plenty of great resources online.

Unfortunately, most of these introductions have two fatal flaws: they are long,
and they are old.
The first is anathema to newcomers---if somebody asks about \LaTeX{},
throwing a 200+ page book at them isn't an encouraging start.
The second matters because of how much the world of typesetting has
changed since 1986.

When \LaTeX{} was first released that year, none of the publishing technologies
we use today existed.
Adobe wouldn't debut their Portable Document Format for seven more years.
Digital printing was a new field, and desktop publishing was a fledgling
curiosity.\punckern\endnote{\textit{Graphic Means:
A History of Graphic Design Production}, directed by Briar Levit (2017)}
This shows---badly---in most \LaTeX{} guides.
If you look for instructions to change your document's font,
you'll get swamped with bespoke nonsense.\punckern\footnote{%
Take all criticisms of \LaTeX's past here with a grain of
salt. After all, the fact that all of the technology around it became
obsolete---multiple times---is a testament to its staying power.}

The good news is that  \LaTeX{} has improved by leaps and bounds in recent years.
It's time for a guide that doesn't weigh you down with decades of legacy
or try (in vain) to be a comprehensive reference.
After all, you're a smart, resourceful individual who knows how to use a search
engine.
This book will:

\begin{enumerate}
\item Teach you the fundamentals of \LaTeX.
\item Point you to places where you can learn more.
\item Show you how to use modern typesetting technologies and techniques,
    like OpenType and microtypography.
\item End promptly thereafter.
\end{enumerate}
\vspace{\baselineskip}

\noindent Let's begin.
