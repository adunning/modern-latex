\chapter{\texorpdfstring{\LaTeX}{LaTeX}?}

\LaTeX{} is a program\footnote{And a markup language,
and (sort of) a programming language, and a bit of a cult.
But we'll get to all of that.}
for creating written documents, such as papers, presentations,
or even the book you're reading right now.
You can view it as an alternative to tools like Microsoft Word,
LiberOffice Writer, Apple Pages, or Google Docs.

\chapter{Typography?}

Modern life is a constant battle for your attention---every day,
dozens of ads, apps, emails, sites, and texts fight
for a few short minutes of your time.
To put your ideas into the world,
nothing is more important than catching and holding
the attention of your audience.
Typography is a tool to do so---a good design doesn't ``look nice''
only for the sake of art---it draws readers in.\punckern\endnote{Matthew Butterick,
\textit{Typography for Docs}, Write The Docs Conference, April 8, 2013}
%Matthew Butterick's \textit{Practical Typography} calls it
%``the visual component of the written word.\quotekern''
It's the \emph{how} of your text.

\begin{adjustwidth}{1.5em}{0pt}
\fontspec{TeX Gyre Termes}\fontsize{12pt}{24pt}\selectfont\raggedright
\noindent Typography is why all the terrible essays you wrote in middle school
looked just like this.
Would you enjoy reading a whole book that looked this way?
\end{adjustwidth}
\bigskip

\noindent It is why
\bigskip

{\fontspec{FuturaCon-ExtBol}\Large DO IT LATER}
\bigskip

\noindent might remind you of a certain shoe company's advertising,
or how

\begin{center}
\fontspec[Ligatures=TeX, Scale=MatchLowercase]{Futura-Med}
\noindent MEN FROM THE PLANET EARTH \\
FIRST SET FOOT UPON THE MOON \\
JULY 1969, A.~D.
\end{center}

\noindent Typography matters: it tells people
what you have to say before they read a single word.

\chapter{Another Guide?}
