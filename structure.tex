\chapter{Document Structure}
\label{structure}

Every \LaTeX{} document is different,
but all share a few common elements.

\section{Packages and the preamble}
In the last chapter, you saw:
\begin{leftfigure}
\begin{lstlisting}
\documentclass{article}

\begin{document}
Hello, World!
\end{document}
\end{lstlisting}
\end{leftfigure}
The area between the \verb|\documentclass| command and the start of the
\texttt{document} environment is called the \introduce{preamble}.
Here, we perform any setup we need---such as importing packages
and defining commands---to control how our document
will look.

As mentioned in \chapref{installation},
\introduce{packages} contain code to modify your document in interesting ways.
The ones found in your \LaTeX{} distribution come from the Comprehensive \TeX{}
Archive Network, or \acronym{ctan},\punckern\footnote{Curious readers may
be wondering what \TeX{} is, and how it differs from \LaTeX.
The short version is that \TeX{} is the typesetting system that \LaTeX{}
is built on top of---the latter is a framework of commands for the former.
(For example, \texttt{\textbackslash documentclass} and friends are provided by
\LaTeX{}, but the \TeX{} engine is what's actually laying out your document.)
The long version is at the back of this book as Appendix~\ref{history}.
We won't discuss how to use plain \TeX{} here. That's for another book---the
\TeX book.}
at \https{ctan.org}.
Package manuals are also found there,
so make it your first stop when learning how to use one.

To import a package, add a \verb|\usepackage| command,
with the package's name as its argument.
As a simple example, let's write a document with the \texttt{metalogo}
package, which adds \verb|\LuaLaTeX| and \verb|\XeLaTeX|:
\begin{leftfigure}
\begin{lstlisting}
\documentclass{article}

\usepackage{metalogo}

\begin{document}
\XeLaTeX{} and \LuaLaTeX{} are neat.
\end{document}
\end{lstlisting}
\end{leftfigure}
\begin{samepage}
should get you a \textsc{pdf} that reads
\begin{leftfigure}
\lm \XeLaTeX{} and \LuaLaTeX{} are neat.
\end{leftfigure}
\end{samepage}
Like other commands, \verb|\usepackage| accepts optional arguments.
The \texttt{geometry} package, for instance,
takes your desired paper size and margins.
For \acronym{us}~\textsc{l}etter paper with one-inch margins,
you type:
\begin{leftfigure}
\begin{lstlisting}
\usepackage[letterpaper,
            left=1in, right=1in, top=1in, bottom=1in
           ]{geometry}
\end{lstlisting}
\end{leftfigure}
Notice
that command arguments can be spaced however you like,
so long as there are no empty lines between arguments.

\section{Titles, sections, and other hierarchy}

Authors often split their works into sections to help readers navigate them.
\LaTeX{} offers seven different commands to break up your documents:
\verb|\part|, \verb|\chapter|, \verb|\section|, \verb|\subsection|,
\verb|\subsubsection|, \verb|\paragraph|, and \verb|\subparagraph|.
To use one, issue the command where you want that section to start,
providing its name as the argument.
For example,
\begin{leftfigure}
\begin{lstlisting}
\documentclass{book}

\begin{document}
\chapter{The Start}
This is a very short chapter in a very short book.

\chapter{The End}
Is the book over yet?

\section{No!}
There's some more we must do before we go.

\section{Yes!}
Goodbye!
\end{document}
\end{lstlisting}
\end{leftfigure}
Some levels are only available in certain document classes---chapters,
for example, only appear in books.
And don't go too crazy with these commands.
Most works only benefit from a few levels of hierarchy.

Sections\footnote{By this I mean all levels, not just
\texttt{\textbackslash section}.}
are automatically numbered---for example,
the title of this chapter was produced with \verb|\chapter{Document Structure}|,
and \LaTeX{} figured out that it was chapter~\ref{structure}.

\section{What next?}

As promised, this book isn't a comprehensive reference,
but it \emph{will} point you to places where you can learn more.
We'll wrap up most chapters with a list of related topics you could
explore next.

Consider learning how to:
\begin{itemize}
\item Automatically start your document with its title, the author's name,
    and the date using \verb|\maketitle|.
\item Have \LaTeX{} build you a table of contents
    with \verb|\tableofcontents|.
\item Control section numbers with \verb|\setcounter{secnumdepth}|
and starred section commands, e.g., \verb|\subsection*{foo}|.
\item Create auto-updating cross-references with \verb|\label| and \verb|\ref|.
\item Use KOMA~Script, a set of document classes and packages
that make it easy to customize nearly every aspect of your document,
from heading fonts to footnotes.
\item Include images with the \texttt{graphicx} package.
\item Add hyperlinks to your \acronym{pdf} with the \texttt{hyperref} package.
\item Split large documents into multiple files using \verb|\input|.
\end{itemize}
